%SP, 31/03/10

\documentclass[aps,article]{revtex4}
\usepackage{epsfig}% Include figure files
\usepackage{dcolumn}% Align table columns on decimal point
\usepackage{bm}% bold math   % for math
\usepackage{amssymb}
\usepackage{amsmath}



\begin{document}






\begin{center}
\Large{\Huge  Equilibrium and Transport Properties of the Hot and Dense Matter created in High-Energy Nuclear Collisions\\}
\large{\Large  Jacquelyn Noronha-Hostler\\}
\today
\end{center}






\section{Ideal Fluid}


In ideal relativistic hydrodynamics (much of the following on ideal fluids is taken from \cite{Romatschke:2009im}) the conservation of energy-momentum is expressed as
\begin{equation}\label{eqn:idealconsv}
\partial_{\mu}T^{\mu\nu}_{0}=0
\end{equation}
where the energy momentum tensor is 
\begin{equation}
T^{\mu\nu}_{0}=\epsilon\, u^{\mu}u^{\nu}-p\Delta^{\mu\nu}
\end{equation}
and $\epsilon$ is the energy density, $p$ is the pressure, and the projector onto to the space perpendicular to $u^\mu$, $\Delta^{\mu\nu}$, is defined as
\begin{equation}
\Delta^{\mu\nu} \equiv g^{\mu\nu}-u^{\mu}u^{\nu}
\end{equation}
where $u_{\mu}\Delta^{\mu\nu}=0$ and the flow velocity $u^\mu=\gamma(1,\vec{v})$ obeys the constraint $u^{\mu}u_{\mu}=1$. 
Out flat space metric is $g^{\mu\nu}=diag(+,-,-,-)$. It is convenient to separate the energy-momentum conservation equations for this relativistic ideal fluid in terms of an equation in the direction parallel to the flow velocity 
\begin{equation}\label{eqn:idealpar}
\boxed{u_{\mu} \partial_{\nu}T^{\mu\nu}_{0}=D\epsilon+(\epsilon +p)\partial_{\mu}u^{\mu}=0}
\end{equation}
where $D=u^{\beta}\partial_{\beta}$ is the comoving fluid derivative and another one perpendicular to the fluid velocity
\begin{equation}\label{eqn:idealper}
\boxed{\Delta_{\nu}^{\alpha}\partial_{\mu}T^{\mu\nu}_{0}=(\epsilon +p)Du^{\alpha}-\nabla^{\alpha}p}
\end{equation}
where $\nabla^{\alpha}=\Delta^{\mu\alpha}\partial_{\mu} $.   

\subsection{Hyperbolic coordinates}

Cartesian coordinates i.e. $t$, $x$, $y$, and $z$ are not the most useful in ultrarelativistic heavy ion collisions due to the approximate boost invariance of the dynamics after the collisions.  Thus, it is useful to rewrite Eqs. (\ref{eqn:idealpar}-\ref{eqn:idealper}) in hyperbolic coordinates where the coordinates are $\tau$, $\xi$, $x$, and $y$ (the beam is in the $z$ direction) where the proper time is
\begin{equation}
\tau=\sqrt{t^2-z^2}
\end{equation}
and the space-time rapidity is
\begin{equation}
\xi=tanh^{-1} \left(\frac{z}{t}\right)=\frac{1}{2}\ln\left(\frac{t+z}{t-z}\right).
\end{equation}
and the metric is
\begin{equation}
g_{\mu\nu}=diag\left(1,-1,-1,-\tau^2\right),
\end{equation}
which is no longer uniform.  Because the metric is no longer uniform one must take into account the Christoffel symbols when performing derivatives, which in this case are called covariant derivatives. Given the metric in hyperbolic coordinates defined above, the only non-zero components of the Christoffel symbols are
\begin{eqnarray}
\Gamma^{\xi}_{\xi\tau}&=&\frac{1}{\tau}\nonumber\\
\Gamma^{\tau}_{\xi\xi}&=&\tau
\end{eqnarray}
so that the covariant gradient of the flow velocity becomes
\begin{equation}\label{eqn:covardev}
\nabla_{\alpha}u^{\nu}=\partial_{\alpha}u^{\nu}+\Gamma^{\nu}_{\alpha\beta}u^{\beta}
\end{equation}
and the covariant derivative perpendicular to the flow is
\begin{equation}
\nabla^{\alpha}_{\perp}=\Delta^{\alpha\mu}\nabla_{\mu}
\end{equation}
and in this case
\begin{equation}\label{eqn:dhyper}
D=u^{\mu}\nabla_{\mu}
\end{equation}
The variables in terms of the new coordinates are
\begin{eqnarray}\label{eqn:othershyper}
\epsilon&=&\epsilon\left(\tau,\xi,x,y\right)\nonumber\\
u^{\mu}&=&u^{\mu}\left(\tau,\xi,x,y\right)\nonumber\\
&\dots&
\end{eqnarray}

\subsection{Bjorken expansion in 1+1 dimensions}
If we assume Bjorken's idea of "boost-invariance" and assume no dynamics in the transverse plane 
\begin{equation}
u^x=u^y=0
\end{equation}
and
\begin{equation}
u^z=\frac{z}{\tau}.
\end{equation}
These all translate into the following relations in hyperbolic coordinates 
\begin{eqnarray}
u^{\xi}&=&-u^t\frac{sinh \xi}{\tau}+u^z\frac{cosh \xi}{\tau}=0 \nonumber \\ 
u^{\tau}=1
\end{eqnarray}
so that the variables $\epsilon$, $p$, $u^{\mu}$, only depend on $\tau$ and, therefore,
\begin{eqnarray}
\epsilon&=&\epsilon\left(\tau\right)\nonumber\\
u^{\mu}&=&\left(1,\vec{0}\right)\nonumber\\
&\dots&
\end{eqnarray}
The expansion rate defined using Eq.\ (\ref{eqn:covardev}) becomes
\begin{equation}\label{eqn:bjrokencovardev}
\nabla_{\mu}u^{\mu}=\partial_{\mu}u^{\mu}+\Gamma^{\xi}_{\xi\tau}u^{\tau}=\frac{1}{\tau}\neq 0
\end{equation}
and is non-vanishing even though $u^{\mu}$ is constant. In this simple case,  the initial conditions of hydrodynamics at a given initial time $\tau=\tau_0$ are completely specified by the initial energy density $\epsilon(\tau_0)$ so that in ideal hydrodynamics one finds
\begin{equation}
D\epsilon+\left(\epsilon+p\right)\nabla_{\mu}u^{\mu}=\partial_{\tau}\epsilon+\frac{\epsilon+p}{\tau}=0.
\end{equation}
If we have a conformal theory in 4 spacetime dimensions the speed of sound is $c_s^{2}=1/3$
\begin{equation}
c_s^2=\frac{dp}{d\epsilon}\rightarrow \epsilon=3p
\end{equation}
then 
\begin{equation}
\frac{d\tau}{\tau}=-\frac{d\epsilon}{\epsilon+p}=-\frac{d\epsilon}{4/3\epsilon}.  
\end{equation}
after integration
\begin{equation}
\epsilon(\tau)=\epsilon(\tau_0)\left(\frac{\tau_0}{\tau}\right)^{4/3}.
\end{equation}
Note that in a conformal plasma, $\epsilon(\tau)=k T^4(\tau)$, where $k$ is a constant that basically counts the number of degrees of freedom. Using the equation above, we see that $T(\tau)=T_0 (\tau_0/\tau)^{1/3}$, where $T_0 = (\epsilon(\tau_0)/k)^{1/4}$.

\section{Viscous Corrections}

If one takes into account dissipative viscous effects then another term must be added to the ideal energy momentum tensor and the total energy momentum tensor becomes 
\begin{equation}\label{eqn:tmunuvisc}
T^{\mu\nu}=T^{\mu\nu}_{0}+\Pi^{\mu\nu}.
\end{equation}
We also define
\begin{equation}
\Delta^{\mu\nu\alpha\beta}\equiv \frac{1}{2}\left[\Delta^{\mu\alpha}\Delta^{\nu\beta}+\Delta^{\mu\beta}\Delta^{\nu\alpha}-\frac{2}{3}\Delta^{\mu\nu}\Delta^{\alpha\beta}\right]\label{eqn:delta}
\end{equation}
which gives the traceless, symmetric projection of a given tensor.  In order to use Eq.\ (\ref{eqn:delta}) it's important to show that $\Delta_{\alpha\mu}\Pi^{\mu\nu}=\Pi^{\nu}_{\alpha}$:
\begin{eqnarray}
\Delta_{\alpha\mu}\Pi^{\mu\nu}&=&g_{\alpha\mu}\Pi^{\mu\nu}-u^{\alpha}\underbrace{u^{\mu}\Pi^{\mu\nu}}_\textrm{=0}\nonumber\\
&=&\Pi^{\nu}_{\alpha}\label{eqn:deltaonpi}
\end{eqnarray}
where $u^{\mu}\Pi^{\mu\nu}=0$ because we use the Landau-Lifshitz frame where all the momentum density is due to the flow of the energy density such that
\begin{equation}
u_{\mu}T^{\mu\nu}=\epsilon u^{\nu}\rightarrow u_{\mu}\Pi^{\mu\nu}=0.
\end{equation}
Applying $\Delta_{\mu\nu}^{\alpha\beta}$ to the viscous stress tensor, it returns only the traceless part of Eq.\ (\ref{eqn:pisep})
\begin{eqnarray}
\Delta^{\alpha\beta}_{\mu\nu}\Pi^{\mu\nu}&=&\frac{1}{2}\left[\Delta^{\alpha}_{\mu}\Delta^{\beta}_{\nu}+\Delta^{\beta}_{\mu}\Delta^{\alpha}_{\nu}-\frac{2}{3}\Delta_{\mu\nu}\Delta^{\alpha\beta}\right]\Pi^{\mu\nu}\nonumber\\
&=&\Pi^{\alpha\beta}-\Delta^{\alpha\beta}\frac{\Pi^{\mu}_{\mu}}{3}\label{eqn:tracelessez}
\end{eqnarray}
Let us then define the result of Eq.\ (\ref{eqn:tracelessez}) as
\begin{equation}
\pi^{\alpha\beta}\equiv \Pi^{\alpha\beta}-\Delta^{\alpha\beta}\frac{\Pi^{\mu}_{\mu}}{3}
\end{equation}
and we can define the term with a trace as 
\begin{equation}
\Pi\equiv -\frac{\Pi^{\mu}_{\mu}}{3}
\end{equation}
so that we can then separate the viscous stress tensor into its traceless part, $\pi^{\alpha\beta}$, and its part with a non-vanishing trace, $\Delta^{\alpha\beta}\Pi$,
\begin{equation}\label{eqn:pisep}
\boxed{\Pi^{\alpha\beta}=\pi^{\alpha\beta}-\Delta^{\alpha\beta}\Pi}\;.
\end{equation}
Note that
\begin{equation}
g_{\mu\nu}\pi^{\mu\nu}=0
\end{equation}
because $\pi^{\mu\nu}$ is traceless.
Furtheremore, we establish the following relationships 
\begin{eqnarray}
\Delta_{\alpha\beta}\Delta^{\alpha\beta}&=&3\label{eqn:pf1}\\
\Delta_{\alpha\beta}\Delta^{\beta\nu}&=&\Delta_{\alpha}^{\nu}\label{eqn:pf2}\\
\Delta^{\mu\nu}_{\alpha\beta}\Delta_{\nu\rho}^{\alpha\beta}&=&\frac{5}{3}\Delta^{\mu}\label{eqn:pf3}\\
\partial^{\mu}-\nabla^{\mu}&=&u^{\mu}D\label{eqn:pf4}
\end{eqnarray}
 in Appendix \ref{app:imptrel}.


In order to obtain the equations of motion,  let's look at what the parallel projection in the flow direction does specifically on the viscous stress tensor
\begin{eqnarray}
u_{\nu}\partial_{\mu}\Pi^{\mu\nu}&=&\partial_{\mu}\underbrace{\left(u_{\nu}\Pi^{\mu\nu}\right)}_\textrm{=0}-\Pi^{\mu\nu}\partial_{\mu}u_{\nu}\nonumber\\
&=&-\Pi^{\mu\nu}\partial_{\mu}u_{\nu}\nonumber\\
&=&-\Pi^{\mu\nu}\left(u_{\mu}D+\nabla_{\mu}\right)u_{\nu}\nonumber\\
&=&-\underbrace{u_{\mu}\Pi^{\mu\nu}}_\textrm{=0}Du_{\nu}-\Pi^{\mu\nu}\nabla_{\mu}u_{\nu}\nonumber\\
&=&-\Pi^{\mu\nu}\nabla_{\mu}u_{\nu}\label{eqn:parsub}
\end{eqnarray}
Any tensor can be divided into its symmetric $(\dots)$ and anti-symmetric $\{\dots\}$ part
\begin{eqnarray}\label{eqn:symnanti}
A_{\mu}B_{\nu}&=&\frac{1}{2}\left(A_{\mu}B_{\nu}+A_{\nu}B_{\mu}\right)+\frac{1}{2}\left(A_{\mu}B_{\nu}-A_{\nu}B_{\mu}\right)\nonumber\\
&=&A_{\left(\mu\right.}B_{\left.\nu\right)}+A_{\left\{\mu\right.}B_{\left.\nu\right\}}.
\end{eqnarray}
Because $\Pi^{\mu\nu}$ is symmetric then the anti-symmetric part of $\nabla_{\perp\mu}u_{\nu}$ can be ignored so we can rewrite Eq.\ (\ref{eqn:parsub}) as
\begin{eqnarray}
u_{\nu}\partial_{\mu}\Pi^{\mu\nu}&=&-\Pi^{\mu\nu}\nabla_{(\mu}u_{\nu)}\label{eqn:parsub2}.
\end{eqnarray}

If one were to look at $\nabla_{\mu}u_{\nu}$ on its own, one could find
\begin{eqnarray}
\partial_{\mu}u_{\nu}&=&u_{\mu}Du_{\nu}+\nabla_{\mu}u_{\nu}\nonumber\\
&=&\frac{1}{2}\left(u_{\mu}Du_{\nu}+\nabla_{\mu}u_{\nu}+u_{\nu}Du_{\mu}+\nabla_{\nu}u_{\mu}\right)+\frac{1}{2}\left(u_{\mu}Du_{\nu}+\nabla_{\mu}u_{\nu}-u_{\nu}Du_{\mu}-\nabla_{\nu}u_{\mu}\right)\nonumber\\
&=&u_{\mu}Du_{\nu}+\frac{1}{2}\left(\nabla_{\mu}u_{\nu}+\nabla_{\nu}u_{\mu}\right)+\frac{1}{2}\left(\nabla_{\mu}u_{\nu}-\nabla_{\nu}u_{\mu}\right)\nonumber\\
&=&u_{\mu}Du_{\nu}+\sigma_{\mu\nu}+\frac{1}{3}\Delta_{\mu\nu}\Theta+\Omega_{\mu\nu}
\end{eqnarray}
where
\begin{eqnarray}
\sigma_{\mu\nu}&\equiv&\frac{1}{2}\left(\nabla_{\mu}u_{\nu}+\nabla_{\nu}u_{\mu}-\frac{2}{3}\Delta_{\mu\nu}\Theta\right)\nonumber\\
&=&\nabla_{(\mu}u_{\nu)}-\frac{1}{3}\Delta_{\mu\nu}\Theta\label{eqn:sigmdef}\\
\Theta&\equiv&\partial_{\mu}u^{\mu}\\
\Omega_{\mu\nu}&\equiv&\frac{1}{2}\left(\nabla_{\mu}u_{\nu}-\nabla_{\nu}u_{\mu}\right)
\end{eqnarray}
where the shear stress tensor $\sigma_{\mu\nu}$ is symmetric, the expansion rate $\Theta$ is symmetric, and the vorticity $\Omega_{\mu\nu}$ is antisymmetric. The term $u_{\mu}Du_{\nu}$ is perpendicular to the flow so it does not play a role when $\Pi^{\mu\nu}$ is contracted with $\partial_{\mu}u_{\nu}$.
Returning to Eq.\ (\ref{eqn:delta}), we can generalize its affect on a given tensor $A_{\alpha\beta}$ as
\begin{equation}\label{eqn:tracelessproj}
A^{\langle\mu\nu\rangle}= \Delta^{\mu\nu\alpha\beta}A_{\alpha\beta}.
\end{equation} 
where $A^{\langle\mu\nu\rangle}$ is the traceless part of $A_{\alpha\beta}$ and the angular brakets $\langle \dots\rangle$ imply a traceless, symmetric tensor.
So we can define
\begin{equation}\label{eqn:nabperp}
\nabla^{\langle\mu}u^{\nu\rangle}\equiv2\sigma^{\mu\nu}=2\nabla^{(\mu}u^{\nu)}-\frac{2}{3}\Delta^{\mu\nu}\Theta.
\end{equation}
We can also show that
\begin{eqnarray}
g_{\mu\nu}\nabla^{\mu}u^{\nu}&=&g_{\mu\nu}\Delta^{\mu\alpha}\partial_{\alpha}u^{\nu}\nonumber\\
&=&\Delta^{\mu\alpha}\partial_{\alpha}u_{\mu}\nonumber\\
&=&g^{\mu\alpha}\partial_{\alpha}u_{\mu}-u^{\alpha}\underbrace{u^{\mu}\partial_{\alpha}u_{\mu}}_\textrm{=0}\nonumber\\
&=&\partial^{\mu}u_{\mu}
\end{eqnarray}
where $u^{\mu}\partial_{\alpha}u_{\mu}=0$ because $u^{\mu}u_{\mu}=1$ so $\partial_{\alpha}\left(u^{\mu}u_{\mu}\right)=0$, then
\begin{eqnarray}
\partial_{\alpha}\left(u^{\mu}u_{\mu}\right)=u^{\mu}\partial_{\alpha}u_{\mu}+u_{\mu}\partial_{\alpha}u^{\mu}
\end{eqnarray}
so $u^{\mu}\partial_{\alpha}u_{\mu}=0$ must hold.
Then, 
\begin{equation}
\Theta=\partial_{\mu}u^{\mu}=\nabla_{\mu}u^{\mu}=\Delta^{\mu\alpha}\partial_{\alpha}u_{\mu}
\end{equation}

Returning to Eq.\ (\ref{eqn:parsub2}) we find
\begin{eqnarray}
u_{\nu}\partial_{\mu}\Pi^{\mu\nu}&=&-\Pi^{\mu\nu}\nabla_{(\mu}u_{\nu)}\nonumber\\
&=&-\Pi^{\mu\nu}\left(\sigma_{\mu\nu}+\frac{1}{3}\Delta_{\mu\nu}\Theta\right)
\end{eqnarray}

%\begin{equation}\label{eqn:shearviscpre}
%\boxed{D\epsilon+\left(\epsilon+p\right)\Theta=\pi^{\mu\nu}\sigma_{\mu\nu}-\Pi\,\Theta}
%\end{equation}
%\begin{equation}\label{eqn:bulkviscpre}
%\boxed{\left(\epsilon+p\right) D u^{\mu}-\nabla^{\mu}_{\perp}p+\Delta^{\mu}_{\nu}\nabla_{\alpha}\Pi^{\alpha\nu}=0}.
%\end{equation}

We can also look at the projection in the perpendicular direction 
\begin{equation}
\Delta_{\nu}^{\mu}\partial_{\alpha}\Pi^{\alpha\nu}=\left(\epsilon+p\right) D u^{\mu}-\nabla^{\mu}p+\Delta^{\mu}_{\nu}\partial_{\alpha}\Pi^{\alpha\nu}=0
\end{equation}
Thus, we end up with the follow equations of motion
\begin{equation}\label{eqn:shearviscpre}
\boxed{D\epsilon+\left(\epsilon+p\right)\partial_{\mu}u^{\mu}-\Pi^{\mu\nu}\sigma_{\mu\nu}+\frac{1}{3}\Pi^{\mu\nu}\Delta_{\mu\nu}\Theta=0}
\end{equation}
\begin{equation}\label{eqn:bulkviscpre}
\boxed{\left(\epsilon+p\right) D u^{\mu}-\nabla^{\mu}p+\Delta^{\mu}_{\nu}\partial_{\alpha}\Pi^{\alpha\nu}=0}.
\end{equation}

We define the entropy density current as 
\begin{equation}\label{eqn:entropy4flux}
s^{\mu}=s\,u^{\mu}
\end{equation}
We know from thermodynamics that
\begin{eqnarray}
\epsilon+p&=&Ts\nonumber\\
Tds&=&d\epsilon.
\end{eqnarray}
Moreover, using the entropy four-flux Eq.\ (\ref{eqn:entropy4flux}) and taking the derivative, we have
\begin{eqnarray}
\partial_{\mu} s^{\mu}&=&u^{\mu}\partial_{\mu}s+s\partial_{\mu}u^{\mu}\nonumber\\
&=&Ds+s\partial_{\mu}u^{\mu}\nonumber\\
&=&\frac{1}{T}\left[D\epsilon+\right(\epsilon+p\left)\partial_{\mu}u^{\mu}\right].
\end{eqnarray}
Recalling Eq.\ (\ref{eqn:shearviscpre}), we find
\begin{equation}\label{eqn:boxent}
\boxed{\partial_{\mu} s^{\mu}=\frac{1}{T}\Pi^{\mu\nu}\nabla_{(\mu}u_{\nu)}}.
\end{equation}
In order for the second law of thermodynamics to hold, $\partial_{\mu} s^{\mu}\geq0$, so $\frac{1}{T}\Pi^{\mu\nu}\nabla_{(\mu}u_{\nu)}\geq0$. We can now subsitute in the split version of $\Pi^{\mu\nu}$ as shown in Eq.\ (\ref{eqn:pisep}) 
\begin{eqnarray}\label{eqn:pinab}
\Pi^{\mu\nu}\nabla_{(\mu}u_{\nu)}&=&\pi^{\mu\nu}\nabla_{(\mu}u_{\nu)}-\Delta^{\mu\nu}\Pi^{\mu\nu}\nabla_{(\mu}u_{\nu)}.
\end{eqnarray}
We can then calculate $\pi^{\mu\nu}\nabla_{(\mu}u_{\nu)}$ using the definition found in Eq.\ (\ref{eqn:nabperp})
\begin{eqnarray}
\pi^{\mu\nu}\nabla_{(\mu}u_{\nu)}&=&\pi^{\mu\nu}\sigma_{\langle\mu}+\frac{1}{3}\Delta_{\mu\nu}\pi^{\mu\nu}\Theta\nonumber\\
&=&\pi^{\mu\nu}\sigma_{\langle\mu}+\frac{1}{3}(\underbrace{g_{\mu\nu}\pi^{\mu\nu}}_\textrm{=0}-u_{\nu}\underbrace{u_{\mu}\pi^{\mu\nu}}_\textrm{=0})\Theta\nonumber\\
&=&\pi^{\mu\nu}\sigma_{\langle\mu}
\end{eqnarray}
and can calculate $\Delta^{\mu\nu}\Pi^{\mu\nu}\nabla_{(\mu}u_{\nu)}$ using Eq.\ (\ref{eqn:symnanti})
\begin{eqnarray}
\Delta^{\mu\nu}\Pi\nabla_{(\mu}u_{\nu)}&=&\frac{1}{2}\Delta^{\mu\nu}\Pi\left(\nabla_{\mu}u_{\nu}+\nabla_{\nu}u_{\mu}\right)\nonumber\\
&=&\frac{1}{2}\Pi\left(\nabla^{\nu}u_{\nu}+\nabla^{\mu}u_{\mu}\right)\nonumber\\
&=&\Pi\nabla^{\mu}u_{\mu}
\end{eqnarray}
so Eq.\ (\ref{eqn:pinab})
\begin{eqnarray}
\Pi^{\mu\nu}\nabla_{(\mu}u_{\nu)}&=&\frac{1}{2}\pi^{\mu\nu}\nabla_{\langle\mu}u_{\nu\rangle}-\Pi\nabla^{\mu}u_{\mu}\\
&=&\pi^{\mu\nu}\sigma_{\mu\nu}-\Pi\Theta
\end{eqnarray}
Then, Eq.\ (\ref{eqn:boxent}) becomes
\begin{equation}\label{eqn:boxent}
\partial_{\mu} s^{\mu}=\frac{1}{2T}\pi^{\mu\nu}\nabla_{\langle\mu}u_{\nu\rangle}-\frac{1}{T}\Pi\nabla^{\mu}u_{\mu}\geq 0
\end{equation}
and we can also rewrite the equation of motion Eq.\ (\ref{eqn:shearviscpre})
\begin{equation}\label{eqn:shearviscallin}
\boxed{D\epsilon+\left(\epsilon+p\right)\partial_{\mu}=\pi^{\mu\nu}\sigma_{\mu\nu}-\Pi\Theta}
\end{equation}

\subsection{Acausality of the Relativistic Version of Navier-Stokes Theory}\label{sec:acausality}
If we assume
\begin{eqnarray}\label{eqn:etazeta}
\pi^{\mu\nu}&=&2\eta\sigma^{\mu\nu}\nonumber\\
\Pi&=&-\zeta\nabla_{\alpha}u^{\alpha}
\end{eqnarray}
where $\eta$ is the shear viscosity and $\zeta$ is the bulk viscosity, then the second law of thermodynamics implies that $\eta>0$  and $\zeta>0$ must hold. In this case, one obtains equations of motion that are similar to the diffusion equation, which is known to have problems with causality. Clearly, in a non-relativistic setting, the lack of causality in the Navier-Stokes equations is not a problem (moreover, the non-relativistic Navier-Stokes equations are stable around hydrostatic equilibrium). 

For relativistic fluids, however, causality must be preserved. The assumptions in Eq. (\ref{eqn:etazeta}) lead to acausal behavior because it allows for modes (with large momentum) that propagate faster than the speed of light. One can show that, in a relativistic theory, acausality leads to instabilities. Thus, one cannot solve the relativistic version of Navier-Stokes equations numerically. 

In order to show this, let's consider small perturbations of the energy density and fluid velocity in a system that is initially in equilibrium and at rest such that
\begin{eqnarray}
\epsilon&=&\epsilon_0+\delta\epsilon(t,x)\nonumber\\
u^{\mu}&=&(1,\vec{0})+\delta u^{\mu}(t,x)\nonumber\\
p&=&p_0+\delta p(t,x)\nonumber\\
\eta&=&\eta_0+\delta \eta(t,x)\nonumber\\
\zeta&=&\zeta_0+\delta \zeta(t,x).
\end{eqnarray}
Returning to Eq.\ (\ref{eqn:bulkviscpre}) and taking just the direction $\mu=y$,
\begin{eqnarray}
\left(\epsilon+p\right) D u^y-\nabla^y p+\Delta^y_{\nu}\partial_{\alpha}\Pi^{\alpha\nu}&=&\left[\epsilon_0+\delta\epsilon(t,x)+p_0+\delta p(t,x)\right] \partial_t \delta u^y(t,x)-\nabla^y \left[p_0+\delta p(t,x)\right]+\Delta^y_{\nu}\partial_{\alpha}\Pi^{\alpha\nu}\nonumber\\
&=&\left[\epsilon_0+p_0\right] \partial_t \delta u^y(t,x)+\underbrace{\left[\delta\epsilon(t,x)+\delta p(t,x)\right] \partial_t \delta u^y(t,x)}_\textrm{non-linear}-\underbrace{\nabla^y \left[p_0+\delta p(t,x)\right]}_\textrm{=0}+\Delta^y_{\nu}\partial_{\alpha}\Pi^{\alpha\nu}\nonumber\\
&=&\left[\epsilon_0+p_0\right] \partial_t \delta u^y(t,x)+\Delta^y_{\nu}\partial_{\alpha}\Pi^{\alpha\nu}+\mathcal{O}(\delta^2).\label{eqn:muy}
\end{eqnarray}
Then, let's concentrate just on the viscous term
\begin{eqnarray}
\Delta^y_{\nu}\partial_{\alpha}\Pi^{\alpha\nu}&=&\partial_{\alpha}\Pi^{\alpha y}
\end{eqnarray}
Let's first look at the case when $\alpha=t$. We know already that $u^{\mu}\Pi_{\mu\nu}=0$.  In the local rest frame $u^{\mu}=(1,0)$, so in order for $u^{\mu}\Pi_{\mu\nu}=0$ then $\Pi_{0\nu}=0$ for all $\nu$.  Thus, $\partial_{t}\Pi^{t y}=0$.
Then we can look at the case where $\alpha=x$
\begin{eqnarray}
\partial_{x}\Pi^{xy}&=&\partial_{x}\left(\pi^{xy}-\Delta^{xy}\Pi\right)\nonumber\\
&=&\partial_{x}\left(2\eta\sigma^{xy}+\underbrace{\zeta\Delta^{xy}\nabla_{\alpha}u^{\alpha}}_\textrm{non-linear}\right)\nonumber\\
&=&\partial_{x}\left[\eta\left(\nabla_{x}u_{y}+\underbrace{\nabla_{y}u_{x}}_\textrm{=0}-\underbrace{\frac{2}{3}\Delta_{xy}\nabla_{\alpha}u^{\alpha}}_\textrm{non-linear}\right)\right]\nonumber\\
&=&\partial_{x}\left[\eta\nabla_{x}u_{y}\right]\nonumber\\
&=&\partial_{x}\left[\left(\eta_0+\delta \eta(t,x)\right)\nabla_{x}\delta u_{y}\right]\nonumber\\
&=&\eta_0\partial^2_{x}\delta u_{y}\nonumber\\
\end{eqnarray}
Thus, Eq.\ (\ref{eqn:muy}) becomes
\begin{equation}
\left[\epsilon_0+p_0\right] \partial_t \delta u^y(t,x)+\eta_0\partial^2_{x}\delta u_{y}+\mathcal{O}(\delta^2)=0,
\end{equation}
which can be rearranged into a diffusion-type evolution equation for the perturbation $\delta u^y (t,x)$
\begin{equation}\label{eqn:diffusioneq}
\partial_t\delta u^y-\frac{\eta_0}{\epsilon_0+p_0}\partial^2_{x}\delta u_{y}=\mathcal{O}(\delta^2).
\end{equation}
Then we can solve Eq.\ (\ref{eqn:diffusioneq}) by using a mixed Laplace-Fourier wave ansatz
\begin{equation}
\delta u^y(t,x)=e^{-wt+ikx}f_{w,k}, 
\end{equation}
which we can subsitute into Eq.\ (\ref{eqn:diffusioneq}) to find the dispersion relation
\begin{equation}
w=\frac{\eta_0}{\epsilon_0+p_0}k^2
\end{equation}
where we can then estimate the speed of diffusion as
\begin{equation}\label{eqn:speedofdiff}
v_T(k)=\frac{dw}{dk}=2\frac{\eta_0}{\epsilon_0+p_0}k.
\end{equation}
The problem with Eq.\ (\ref{eqn:speedofdiff}) is that the velocity $v_T$ is linearly dipendent on the wavenumber.  Thus, as $k$ increases $v_T$ will eventually exceed the speed of light, which violates causality.

A way to resolve this problem is to consider the phenomenological equation proposed by Cattaneo (and later by Israel and Stewart) where the shear tensor, $\pi^{\mu\nu}$, relaxes towards its Navier-Stokes limit. For instance, for $\pi^{xy}$ we obtain (in the local rest frame where $u^{\mu}=(1,0)$)
\begin{equation}
\tau_{\pi}\partial_{t}\pi^{xy}+\pi^{xy}=2\,\eta\,\sigma^{xy}
\end{equation}
where $\tau_{\pi}$ is the new transport coefficient known as the relaxation time (it is clear from the equation above that this quantity defines the timescale that it takes for flow gradients to be converted into pressure). However, there are some limitations on $\tau_{\pi}$ and if $\tau_{\pi}\rightarrow 0$ then the problem of acausality arises again. In fact, one can show that the system is causal and stable around hydrostatic equilibrium (where $T^{\mu\nu}=diag(\epsilon,p,p,p)$) if $(\eta/s)/(\tau_{\pi}T) \leq 3(1-c_s^2)/4$ \cite{tomoi}. 

From kinetic theory, one can show that this new transport coefficient is basically the mean free path, i.e, $\tau_\pi \sim \ell_{MFP}$. This shows that $\tau_\pi$ is a microscopic timescale in the dynamics.   

\subsection{Equations from Kinetic Theory}

In order to avoid issues with acausality, Isreal and Steward derived  a set of relativistic fluid-dynamical equations using the 14-moment approximation.  The second moment of the Boltzmann equation is used to extract the equations of motion and to determine the transport coefficents.  This then gives 14 coefficients that are needed to describe the distribution function.  The choice in truncating at the second moment is rather arbitrary and it has actually been shown that the Isreal-Stewart equations are not in good agreement with the numerical soluation of the Boltzmann equation \cite{numboltz}.

The Boltzmann equation
\begin{equation}
K^{\mu}\partial_{\mu}f_k=C[f]
\end{equation}
where $C[f]$ is the collision term where only two-to-two collisions ($C[f]=0$ at equilibrium or when the system is far from equilibrium) are considered
\begin{equation}
C[f]=\frac{1}{2}\int dK^{\prime}dPdP^{\prime}W_{KK^{\prime}\rightarrow PP^{\prime}}\left(f_P f_{P^{\prime}}\tilde{f}_K \tilde{f}_{K^{\prime}}-f_K f_{K^{\prime}}\tilde{f}_P \tilde{f}_{P^{\prime}}\right)
\end{equation}
and $K^{\mu}=\left(E_k,\bf{k}\right)$ is the four-momentum, $E_k=\sqrt{\bf{k}^2+m^2}$, and $f_k\left(x^{\mu},K^{\mu}\right)$ is the one particle distribution function where $\tilde{f}_{K}=1-a f\left(x^{\mu},K^{\mu}\right)$ where a=1 (or -1) for fermions (or bosons) and a=0 for a Boltzmann gas. 
Additionally, the Lorentz-invariant measurement is 
\begin{equation}
dK=g\frac{d^3\vec{K}}{(2\pi)^3 E_{\bf{k}}}.
\end{equation}

We can then calculate the expectation values for the conserved particle current
\begin{equation}\label{eqn:nmuang}
N^{\mu}=\langle K^{\mu}\rangle=\int dK K^{\mu}f_K
\end{equation}
and the energy moment tensor
\begin{equation}\label{eqn:tmunuang}
T^{\mu\nu}=\langle K^{\mu}K^{\nu}\rangle=\int dK K^{\mu}K^{\nu}f_K
\end{equation}
where one notices that $\langle \dots \rangle=\int dK \left(\dots\right) f_K$.

The four-momentum can be decomposed into its perpendicular and parallel to $u^{\mu}$ parts
\begin{equation}\label{eqn:decomp}
K^{\mu}=(u\cdot K)u^{\mu}+K^{\langle \mu\rangle}
\end{equation}
where $A\cdot B=A_{\mu}B^{\mu}$ and $K^{\langle \mu\rangle}$ is similar to Eq.\ (\ref{eqn:tracelessproj}), i.e. $K^{\langle \mu\rangle}=\Delta^{\mu\nu}K_{\nu}$.  Using Eq.\ (\ref{eqn:decomp}), we can then rewrite Eqs. (\ref{eqn:nmuang}-\ref{eqn:tmunuang})
\begin{eqnarray}
N^{\mu}&=& n u^{\mu}+n^{\mu}    \nonumber\\
T^{\mu\nu}&=& \epsilon u^{\mu}u^{nu}-\Delta^{\mu\nu}\left(P_0+\Pi\right)+\pi^{\mu\nu}
\end{eqnarray}
where the particle density n, the particle diffusion current $n^{\mu}$, the energy density $\epsilon$, the shear stress tensor $\pi^{\mu\nu}$, the sum of thermodynamic pressure $P_0$, the bulk viscous pressure $\Pi$ are defined, respectively
\begin{eqnarray}
n&=&\langle u\cdot K\rangle\nonumber\\
n^{\mu}&=&\langle K^{\langle \mu\rangle}\rangle\nonumber\\
\epsilon&=&\langle\left(u\cdot K\right)^2\rangle\nonumber\\
\pi^{\mu\nu}&=&\langle K^{\langle \mu}K^{\nu\rangle}\rangle\nonumber\\
P_0+\Pi&=&-\frac{1}{3}\langle \Delta^{\mu\nu}K_{\mu}K_{\nu}\rangle
\end{eqnarray}
where $P_0$ and all other variables indicated by $_0$ are for the ideal case without viscosity. Then the local distribution function is
\begin{equation}\label{eqn:f0K}
f_{0K}=\left[\exp\left(\beta_0u\cdot K-\alpha_0\right)+a\right]^{-1}
\end{equation}
where $\beta_0=1/T$ and $\alpha_0=\mu_B/T$ (the ratio of the chemical potential to the temperature). Eq.\ (\ref{eqn:f0K}) is then used to find
\begin{eqnarray}
\epsilon=\epsilon_0&=&\langle\left(u\cdot K\right)^2\rangle_0\nonumber\\
n=n_0&=&\langle u\cdot K\rangle_0
\end{eqnarray}
where $\langle \dots \rangle_0=\int dK \left(\dots\right) f_{0K}$.

Then one can derive the difference between the thermodynamic pressure and the bulk viscous pressure so that
\begin{eqnarray}
P_0&=&-\frac{1}{3}\langle\Delta^{\mu\nu}K_{\mu}K_{\nu}\rangle_0\nonumber\\
\Pi&=&-\frac{1}{3}\langle\Delta^{\mu\nu}K_{\mu}K_{\nu}\rangle_{\delta}
\end{eqnarray}
where $\langle \dots \rangle_{\delta}=\langle\dots\rangle-\langle\dots\rangle_0$.

Up until this point no assumptions beyond those behind the validty of the Boltzmann were employed.  



\subsection{Memory Function}
\subsubsection{Bulk Viscosity: 1+1}

Let us first consider only the bulk channel because it has significantly less transport coefficients. In \cite{SPHrio} the equations for SPH for the (1+1) dimension case (where the motion in the transverse direction is ignored and only the longitudinal dynamics are considered) are derived, which we will show here.    As always Eq.\ (\ref{eqn:idealconsv}) is needed for the conservation of the energy-momentum tensor.  Since we are only consider the bulk viscosity Eq.\ (\ref{eqn:tmunuvisc}) can be rewritten as
\begin{equation}
T^{\mu\nu}=\left(\epsilon+p+\Pi\right)u^{\mu}u^{\nu}-\left(p+\Pi\right)g^{\mu\nu}.
\end{equation}

In hydrodynamics, fluid cells have a finite volume $V^*$ at point $\vec{r}$.  The flow of the fluid inside the cell deforms the volume as a function of time so you have something like
\begin{equation}\label{eqn:prevol}
\frac{1}{V^*}\frac{dV^*}{dt}=\nabla\cdot\vec{v}.  
\end{equation}
This is known as the continuity equation and can be expressed in covariant form
\begin{equation}\label{eqn:covform}
\partial_{\mu}\left(\sigma u^{\mu}\right)=0
\end{equation}
where $\sigma$ is the proper reference density 
\begin{equation}
\sigma=\frac{1}{V}=\frac{\gamma}{V^*}.
\end{equation}

In the presence of an irreversible curent the entropy is no longer conserved, rather
\begin{equation}\label{eqn:nonconentr}
\partial_{\mu}s^{\mu}=-\frac{1}{T}\Pi\partial_{\mu}u^{\mu}
\end{equation}
where the entropy four-flux, $s^{\mu}$, is shown in Eq.\ (\ref{eqn:entropy4flux}). Eq.\ (\ref{eqn:nonconentr}) is derived in Eq.\ (\ref{eqn:boxent}). 
Defining the extensive measure of the entropy (i.e. the total entropy) as
\begin{equation}
\tilde{s}=sV=\frac{s}{\sigma},
\end{equation}
we can rewrite Eq.\ (\ref{eqn:nonconentr}) as
\begin{eqnarray}\label{eqn:spart1}
\partial_{\mu}s^{\mu}&=&\partial_{\mu}\left(su^{\mu}\right)\nonumber\\
&=&\partial_{\mu}\left(\sigma\tilde{s}u^{\mu}\right)\nonumber\\
&=&\sigma u^{\mu}\partial_{\mu}\tilde{s}+\tilde{s}\partial_{\mu}\left(\sigma u^{\mu}\right)\nonumber\\
&=&\sigma u^{\mu}\partial_{\mu}\tilde{s}.
\end{eqnarray}
Then, defining the extensive measurement inside the fluid cell of the irreversible current, $\Pi$, as 
\begin{equation}\label{eqn:pitilde}
\tilde{J}=\tilde{\Pi}=\Pi V=\frac{\Pi}{\sigma}
\end{equation}
and
\begin{equation}
\frac{d}{d\tau}=u^{\mu}\partial_{\mu},
\end{equation}
we obtain
\begin{equation}\label{eqn:switch2ex}
T\frac{d\tilde{s}}{d\tau}=-\tilde{J}F=-\tilde{\Pi}\partial_{\mu}u^{\mu}.
\end{equation}
From Eq.\ (\ref{eqn:switch2ex}) we see that in irreversible thermodynamics the net entropy production in the cell is given by the product of the irreversible displacement in each cell and the corresponding thermodyanmic force field in each cell
\begin{equation}\label{eqn:force}
F=\partial_{\alpha}u^{\alpha}.
\end{equation}
Returning to Eq.\ (\ref{eqn:spart1}),
\begin{eqnarray}\label{eqn:spart2}
\partial_{\mu}s^{\mu}&=&\sigma u^{\mu}\partial_{\mu}\tilde{s}\nonumber\\
&=&\sigma u^{\mu}\partial_{\mu}\left(\frac{s}{\sigma}\right)\nonumber\\
&=&\sigma u^{\mu}s\partial_{\mu}\left(\frac{1}{\sigma}\right)+u^{\mu}\partial_{\mu}s\nonumber\\
\end{eqnarray}
and we can also expand the left hand side of the equation so that
\begin{eqnarray}\label{eqn:spart3}
\partial_{\mu}s^{\mu}&=&\partial_{\mu}\left(su^{\mu}\right)\nonumber\\
&=&u^{\mu}\partial_{\mu}\left(s\right)+s\partial_{\mu}\left(u^{\mu}\right)\nonumber\\
\end{eqnarray}
Thus,
\begin{equation}
\partial_{\mu}u^{\mu}=\sigma u^{\mu}\partial_{\mu}\left(\frac{1}{\sigma}\right)=\sigma \frac{d}{d\tau}\left(\frac{1}{\sigma}\right),
\end{equation}
which is our force
\begin{equation}\label{eqn:forcebulk}
F=\partial_{\alpha}u^{\alpha}=\sigma\frac{d}{d\tau}\left(\frac{1}{\sigma}\right).
\end{equation}
The significance of Eq.\ (\ref{eqn:forcebulk}) is that the termodynamic force for the fluid cell is the change of the cell volume, which is exactly the physical meaning of the bulk viscosity i.e. the resistance to change of the volume of hte system.

As discussed previously, in relatistic Navier-Stokes it is assumed that the bulk viscosity per volume element is produced by the thermodynamic force without retardation (see Section \ref{sec:acausality} and specifically Eq.\ (\ref{eqn:etazeta})).  
However, as mentioned previously, there are issues with acausality and instability.  Thus, we introduce a memory effect to the irreversible current by using the simplest  memory function which can be reduced to the differential equation, thus,
\begin{equation}
G\left(\tau,\tau^{\prime}\right)\rightarrow\frac{1}{\tau_R\left(\tau^{\prime}\right)}e^{-\int_{\tau^{\prime}}^{\tau}\frac{1}{\tau_R\left(\tau^{\prime\prime}\right)}d\tau^{\prime\prime}}
\end{equation}
where $\tau_R$ is the relaxation time, which gives the time scale for the retardation, and it is, generally, a function of the local proper time $\tau=\tau({\bf r}, t)$ through the termodynamical quantities. Introducing the memory function in the irreversible current,
\begin{equation}\label{eqn:mfic}
\tilde{\Pi}(\tau)=\int_{-\infty}^{\tau}d\tau^{\prime}G\left(\tau,\tau^{\prime}\right)\frac{\zeta}{\sigma}\partial_{\alpha}u^{\alpha}\left(\tau^{\prime}\right)
\end{equation}

When you start with a finite initial time, $\tau_0$, Eq.\ (\ref{eqn:mfic}) becomes
\begin{equation}\label{eqn:mficIC}
\tilde{\Pi}(\tau)=\int_{\tau_0}^{\tau}d\tau^{\prime}G\left(\tau,\tau^{\prime}\right)\frac{\zeta}{\sigma}\partial_{\alpha}u^{\alpha}\left(\tau^{\prime}\right)+e^{-\left(\tau-\tau_0\right)/\tau_R}\tilde{\Pi}_0
\end{equation}
where $\Pi_0$ is an initial condition give at $\tau_0$. 

We can use the fundamental theorem of calculate to find the memory function from Eq.\ (\ref{eqn:mfic})
\begin{equation}
\frac{d}{d\tau}\int_{\tau_0}^{\tau}d\tau^{\prime}G\left(\tau,\tau^{\prime}\right)X(\tau^{\prime})=G\left(\tau,\tau\right)X(\tau)+\int_{\tau_0}^{\tau}d\tau^{\prime}X(\tau^{\prime})\frac{d}{d\tau}G\left(\tau,\tau^{\prime}\right)
\end{equation}
where $X\left(\tau\right)=\frac{\zeta}{\sigma}\partial_{\alpha}u^{\alpha}\left(\tau\right)$.  Then if we assume that the integral in the exponent of the exponential of the Green's function is $F(\tau)-F(\tau^{\prime})=-\int_{\tau^{\prime}}^{\tau}\frac{1}{\tau_R\left(\tau^{\prime\prime}\right)}d\tau^{\prime\prime}$ we find
\begin{equation}
\frac{d}{d\tau}\int_{\tau_0}^{\tau}d\tau^{\prime}G\left(\tau,\tau^{\prime}\right)X(\tau^{\prime})=\frac{X\left(\tau\right)}{\tau_R (\tau)}+F^{\prime}(\tau)\int_{\tau_0}^{\tau}d\tau^{\prime}X(\tau^{\prime})G\left(\tau,\tau^{\prime}\right)
\end{equation}
where $F^{\prime}(\tau)=-\frac{1}{\tau_R\left(\tau\right)}$ so
\begin{equation}
\frac{d}{d\tau}\int_{\tau_0}^{\tau}d\tau^{\prime}G\left(\tau,\tau^{\prime}\right)X(\tau^{\prime})=\frac{X\left(\tau\right)}{\tau_R (\tau)}-\frac{1}{\tau_R (\tau)}\int_{\tau_0}^{\tau}d\tau^{\prime}X(\tau^{\prime})G\left(\tau,\tau^{\prime}\right),
\end{equation}
which gives
\begin{equation}\label{eqn:mf}
\tilde{\Pi}+\tau_R\frac{d\tilde{\Pi}}{d\tau}=-\frac{\zeta}{\sigma}\partial_{\alpha}u^{\alpha}.
\end{equation}
Substituting in Eq.\ (\ref{eqn:pitilde}) so that we can describe Eq.\ (\ref{eqn:mf}) in terms of the density
\begin{eqnarray}
-\frac{\zeta}{\sigma}\partial_{\alpha}u^{\alpha}&=&\tau_R\frac{d}{d\tau}\left(\frac{\Pi}{\sigma}\right)+\frac{\Pi}{\sigma}\nonumber\\
&=&\tau_R\Pi\frac{d}{d\tau}\left(\frac{1}{\sigma}\right)+\tau_R\frac{1}{\sigma}\frac{d\Pi}{d\tau}+\frac{\Pi}{\sigma}\nonumber\\
&=&\tau_R\frac{\Pi}{\sigma}\partial_{\mu}u^{\mu}+\tau_R\frac{1}{\sigma}\frac{d\Pi}{d\tau}+\frac{\Pi}{\sigma}\nonumber\\
\tau_R\frac{d\Pi}{d\tau}+\Pi&=&-\left(\zeta+\tau_R\Pi\right)\partial_{\mu}u^{\mu}.\label{eqn:mffinitecor}
\end{eqnarray}
Eq.\ (\ref{eqn:mffinitecor}) has an extra term in it compared to the linearized kinetic theory (discussed in \cite{prc2007})
\begin{equation}\label{eqn:nofinitecor}
\tau_R\frac{d\Pi}{d\tau}+\Pi=-\zeta\partial_{\mu}u^{\mu}.
\end{equation}
The difference between Eq.\ (\ref{eqn:mffinitecor}) and Eq.\ (\ref{eqn:nofinitecor}) is that Eq.\ (\ref{eqn:mffinitecor}) takes into account finite size effects whereas Eq.\ (\ref{eqn:nofinitecor}) does not.  The reason Eq.\ (\ref{eqn:mffinitecor})  does not explicitly depend on $1/\sigma$ is because the cell volume is irrelevant because the length is much larger than the mean free path and much smaller than the typical hydrodynamic scale. One interest difference between the two equations is that in Eq.\ (\ref{eqn:nofinitecor}) bulk viscosity scalar is always negative whereas in Eq.\ (\ref{eqn:mffinitecor}) the bulk viscosity scalar can be both negative or positive and has a minimum at
\begin{equation}
\Pi_{min}=-\frac{\zeta}{\tau_R}.
\end{equation}

\subsubsection{Bulk Viscosity: Solving with SPH}
At this moment in time it is difficult to calculate a precise quantitative description of the bulk visocisty.  The Boltzmann equation only is calcuable for two-body collisions and as of yet a consistent multi-body collision equation for non-Newtonian fluids does not exist.  Thus, for now we simply assume that
\begin{equation}
\zeta=as
\end{equation}
where $a$ is a constant and $s$ is the entropy density in the local rest frame.  In order to preseve casaulity the relaxation time cannot go to zero and $\zeta$ and $\tau_R$ are related through the velocity $v=\sqrt{1/b+\alpha}$ where $\alpha=dp/d\epsilon$ and we take $b$ as a constant such that
\begin{equation}
\tau_R=\frac{\zeta}{\epsilon+p}b.
\end{equation}
Thus, we can now speak in terms of the paremeters $a$ and $b$. 

Let us remind ourselves of the equations that need to be solved:
\begin{eqnarray}
D\epsilon+\left(\epsilon+p\right)\Theta&=&-\Pi\,\Theta\label{eqn:s1}\\
\left(\epsilon+p\right) D u^{\mu}&=&\nabla^{\mu}_{\perp}p-\Delta^{\mu}_{\nu}\nabla_{\alpha}\Delta^{\alpha\nu}\Pi\label{eqn:s2}\\
\tau_R\frac{d\Pi}{d\tau}+\Pi&=&-\left(\zeta+\tau_R\Pi\right)\partial_{\mu}u^{\mu}.\label{eqn:s3}
\end{eqnarray}
Here we have eliminated the shear viscous tensor because we are only considering the bulk viscosity.  In order to solve Eqs.\ (\ref{eqn:s1}-\ref{eqn:s3}) numerically we will used smoothed particles hydrodynamics or SPH.  SPH was initially introduced in astrophysics in \cite{astroSPH} and was first extended to heavy-ion collisions in \cite{SPHheavyion}.

The idea behind SPH is to obtain an approximte solution in hydrodynamics by discretizing the fluid into a set of effective particles. It can then be interpreted as a physical model of the collective motion in terms of a finite set of dynamical variables. Consider first a distribution $a\left({\bf r},t\right)$ of an extensive physical quantity $A$. The behavior of $a\left({\bf r},t\right)$ contains the effects of whole microscopic degrees of freedom. Because we are interested in the global behavior of $a\left({\bf r},t\right)$ and how it related to the experimental observables, we introduce a coarse-grain proceedure for $a\left({\bf r},t\right)$.  Thus, we introduce a kernel function, $W({\bf r}-{\bf \tilde{r}},h)$, that maps the original distribution $a$ to a coarse-grained version $a_{CG}$
\begin{equation}\label{eqn:CGint}
a_{CG}\left({\bf r},t\right)=\int a\left({\bf \tilde{r}},t\right)W({\bf r}-{\bf \tilde{r}},h)d{\bf \tilde{r}}
\end{equation}
where 
\begin{equation}\label{eqn:normalizeW}
\int W({\bf r}-{\bf \tilde{r}},h)d{\bf \tilde{r}}={\bf 1}.
\end{equation}
The parameter $h$ represents the width of $W$ and serves as a cut-off parameter for short wavelength modes
\begin{equation}
W({\bf r},h)\rightarrow 0,\;|{\bf r}|\gtrsim h
\end{equation}
such that
\begin{equation}
\lim_{h\rightarrow 0} W({\bf \tilde{r}},h)=\delta({\bf \tilde{r}}).
\end{equation}
Then, $h$ is the typical length scale for the coarse-graining, thus, we will take $h$ as the scale for coarse graining in QCD dynamics (i.e. the mean free path of the partons) to obtain the hydrodynamics of QGP (so $h\approx 0.f\;fm$). 


Next we have to approximate the coarse-graining distribution, $a_{CG}\left({\bf r},t\right)$, by replacing the integral in Eq.\ (\ref{eqn:CGint}) with a summation over a finite and discrete set of points $\{{\bf r}_{\alpha}(t),\alpha=1,\dots,N_{SPH}\}$ where $N_{SPH}$ is the total number of SPH effective particles.  Then we have
\begin{equation}\label{eqn:sphdisc}
a_{SPH}\left({\bf r},t\right)=\sum_{\alpha=1}^{N_{SPH}}A_{\alpha}(t)W(|{\bf r}-{\bf r}_{\alpha}(t)|)
\end{equation} 
If our choice in $\{A_{\alpha}(t),\alpha=1,\dots,N_{SPH}\}$ and $\{{\bf r}_{\alpha}(t),\alpha=1,\dots,N_{SPH}\}$ are appropriate, then Eq.\ (\ref{eqn:sphdisc}) will converge to Eq.\ (\ref{eqn:CGint}).  Therefore, appropriate choices are vital, thankfully, we can determine $\{A_{\alpha}(t),\alpha=1,\dots,N_{SPH}\}$ and $\{{\bf r}_{\alpha}(t),\alpha=1,\dots,N_{SPH}\}$ through the dynamics of the system as we will show in the following. 

Using the example of the contuity equation, which can show that it can be solved very simply using SPH.
First we choose the reference density $\sigma^*$, which must be conserved such that
\begin{equation}\label{eqn:conteq}
\frac{\partial\sigma^*}{\partial t}+\nabla\cdot{\bf j}=0
\end{equation}
where ${\bf j}=\sigma^*{\bf v}$ is the current associated with the density $\sigma^*$. In SPH formalism we can express the current as 
\begin{eqnarray}
\mathbf{j}_{SPH}\left( \mathbf{r},t\right) &=&\sum_{\alpha =1}^{N_{SPH}}{\bf v}_i\nu
_{\alpha }W(|\mathbf{r}-\mathbf{r}%
_{\alpha }(t)|),
\end{eqnarray}
so that
\begin{eqnarray}
\nabla\cdot\mathbf{j}_{SPH}\left( \mathbf{r},t\right) &=&\sum_{\alpha =1}^{N_{SPH}}{\bf v}_i\nu
_{\alpha }\nabla W(|\mathbf{r}-\mathbf{r}%
_{\alpha }(t)|).\label{eqn:nabright}
\end{eqnarray}
The right hand side of Eq.\ (\ref{eqn:conteq}) is, however,
\begin{eqnarray}
\sigma _{SPH}^{\ast }\left( \mathbf{r},t\right)& =&\sum_{\alpha
=1}^{N_{SPH}}\nu _{\alpha }W(|\mathbf{r}-\mathbf{r}_{\alpha }(t)|)\label{eqn:sigmasph}\\
\frac{\partial\sigma _{SPH}^{\ast }\left( \mathbf{r},t\right)}{\partial t}& =&\sum_{\alpha
=1}^{N_{SPH}}\nu _{\alpha }\frac{d}{dt}W(|\mathbf{r}-\mathbf{r}_{\alpha }(t)|)\nonumber\\
& =&\sum_{\alpha
=1}^{N_{SPH}}\nu _{\alpha }\frac{d\mathbf{r}_{\alpha }(t)}{dt}\cdot\nabla W(|\mathbf{r}-\mathbf{r}_{\alpha }(t)|),
\end{eqnarray}
which comparing with Eq.\ (\ref{eqn:nabright}) gives
\begin{equation}
{\bf v}_i=\frac{d\mathbf{r}_{\alpha }(t)}{dt}
\end{equation}
so we see that Eq.\ (\ref{eqn:conteq}) is, indeed, satisfied by SPH.

We can then normalize $W$ (as in Eq.\ (\ref{eqn:normalizeW})) for Eq.\ (\ref{eqn:sigmasph}) because the $v_{\alpha}$'s are constant
\begin{equation*}
\int_{SPH}\sigma ^{\ast }\left( \mathbf{r},t\right) d^{3}\mathbf{r=}%
\sum_{\alpha =1}^{N_{SPH}}\nu _{\alpha }.
\end{equation*}
Then $v_{\alpha}$ can be interpreted as the conserved quantity at the point ${\bf r}={\bf r}_{\alpha}(t)$.  Thus, the distribution $\sigma _{SPH}^{\ast }\left( \mathbf{r},t\right) $ is a summation of piece-wise distributions carrying the density
\begin{equation}
\sigma^{\ast }\left( \mathbf{r}_{\alpha},t\right)=\nu _{\alpha }W(|\mathbf{r}-\mathbf{r}_{\alpha }(t)|)
\end{equation}
for each SPH "particle" and the distribution for each "particle" is
\begin{equation}
a\left( \mathbf{r}_{\alpha},t\right)=A_{\alpha}(t)W(|{\bf r}-{\bf r}_{\alpha}(t)|).
\end{equation}
Then we find that $A_{\alpha}$ is
\begin{equation}\label{eqn:Aalpha}
A_{\alpha }\left( t\right) =\nu _{\alpha }\frac{a(\mathbf{r}_{\alpha },t)}{%
\sigma ^{\ast }(\mathbf{r}_{\alpha },t)}
\end{equation}	
which is the quantity $A$ carried by the SPH particle at ${\bf r}={\bf r}_{\alpha}(t)$. This then changes Eq.\ (\ref{eqn:sphdisc}) into
\begin{equation}\label{eqn:asphsub}
a_{SPH}\left({\bf r},t\right)=\sum_{\alpha=1}^{N_{SPH}}\nu _{\alpha }\frac{a(\mathbf{r}_{\alpha },t)}{%
\sigma ^{\ast }(\mathbf{r}_{\alpha },t)}W(|{\bf r}-{\bf r}_{\alpha}(t)|).
\end{equation}
The total amount of $A$ in the system is then found by doing a summation over Eq.\ (\ref{eqn:Aalpha})
\begin{equation}
A(t)=\sum_{\alpha}^{N_{SPH}}A_{\alpha}(t).
\end{equation}

When you consider an ideal fluid you can take the entropy density as the choosen reference density (because it is conserved) and the dynamics of the parameters $\{{\bf r}_{\alpha}(t),\alpha=1,\dots,N_{SPH}\}$ are determined from the variational principle from the action of ideal hydrodynamics.  However, since we are considering a dissipative fluid, the entropy density is no longer conserved and we instead consider the proper reference density $\sigma$ as show in Eq.\ (\ref{eqn:covform}), which we will use as a reference density for viscous fluids.  Here the four-velocity $u^{\mu}$ is defined in terms of the local rest frame of the energy flow, i.e. the Landau frame.  The proper reference density can then be written in the SPH form
\begin{equation}\label{eqn:prd}
\sigma ^{\ast }(\mathbf{r},t)=\sum_{\alpha =1}^{N_{SPH}}\nu _{\alpha }W(|%
\mathbf{r}-\mathbf{r}_{\alpha }(t)|),
\end{equation}
where 
\begin{equation}
\sigma ^{\ast }=\sigma u^{0}
\end{equation}
is the specific density in the
laboratory frame and 
\begin{equation}
\nu _{\alpha }=\frac{1}{\sigma ^{\ast }(\mathbf{r}_{\alpha},t)}
\end{equation}
is the inverse of the specific volume
of the SPH particle $\alpha ,$ and is chosen as an arbitrary constant. The specific volume is interpretated as the volume of the fluid cell.  However, as we have shown in Eq.\ (\ref{eqn:mffinitecor}) our final results do not depend on the volume so we can simply set $\nu _{\alpha }=1$ for simplicity's sake. For the kernel $W({\bf r})$ we use the spline function. This only works if the lines of flow in spaced defined by the velocity field $u^{\mu}$ do not cross each other during the time evolution, basically, this excludes turbulence and singularities in the flow lines. 

We can now apply SPH to the non-linear memory function in Eq.\ (\ref{eqn:mffinitecor}).  First we can rewrite Eq.\ (\ref{eqn:pitilde}) in the SPH form. We can simply substitute in $\Pi$ for $\sigma ^{\ast }$ in Eq.\ (\ref{eqn:asphsub})
\begin{equation}
\Pi =\sum_{\alpha =1}^{N_{SPH}}\nu _{\alpha }\frac{\Pi _{\alpha }}{\sigma
_{\alpha }^{\ast }}W(|\mathbf{r}-\mathbf{r}_{\alpha }(t)|),
\label{Binterpolation}
\end{equation}%
where the SPH expression for the viscosity, $\Pi_{\alpha}$, is described using the memory function in Eq.\ (\ref{eqn:mf})
\begin{equation}
\gamma _{\alpha }\frac{d\Pi _{\alpha }}{dt}=-\frac{\zeta }{\tau _{R}}\left(
\partial _{\mu }u^{\mu }\right) _{\alpha }-\frac{1}{\tau _{R}}\Pi _{\alpha }
\label{BulkSPH}
\end{equation}%
where we recall that the proper time can be rewritten as $d\tau=dt/\gamma_{\alpha}$ and $\gamma_{\alpha}$ is the Lorentz factor of the $\alpha^{th}$ particle.  Then, we can use the SPH expression for the entropy density, $s^{*}$, which is
\begin{equation}
s^{*}=\sum_{\alpha =1}^{N_{SPH}}\nu _{\alpha }\left(\frac{s}{\sigma
}\right)_{\alpha}W(|\mathbf{r}-\mathbf{r}_{\alpha }(t)|),
\end{equation}
where $s=s^*/\gamma$ and Eq.\ (\ref{eqn:nonconentr}) and Eq.\ (\ref{eqn:spart1}) to find
\begin{eqnarray}
\sigma D\tilde{s}&=&-\frac{1}{T}\Pi\partial_{\mu}u^{\mu}\nonumber\\
\sigma \frac{d\tilde{s}}{d\tau}&=&\nonumber\\
\sigma^* \frac{d\tilde{s}}{dt}&=&\nonumber\\
\frac{d}{dt}\left(\frac{s}{\sigma}\right)&=&-\frac{1}{T}\frac{\Pi}{\sigma^* }\partial_{\mu}u^{\mu}\label{eqn:sphconvert}
\end{eqnarray}
In order to convert Eq.\ (\ref{eqn:sphconvert}) into SPH we can follow these steps for two quantities where $A=B$ 
\begin{eqnarray}
A&=&B\nonumber\\
\sum_{\alpha =1}^{N_{SPH}}\nu _{\alpha }\left(\frac{A}{\sigma
}\right)_{\alpha}W(|\mathbf{r}-\mathbf{r}_{\alpha }(t)|)&=&\sum_{\alpha =1}^{N_{SPH}}\nu _{\alpha }\left(\frac{B}{\sigma
}\right)_{\alpha}W(|\mathbf{r}-\mathbf{r}_{\alpha }(t)|).
\end{eqnarray}
Using this we can then conver Eq.\ (\ref{eqn:sphconvert}) into SPH
\begin{eqnarray}
\frac{d}{dt}\left(\frac{s}{\sigma}\right)&=&-\frac{1}{T}\frac{\Pi}{\sigma^* }\partial_{\mu}u^{\mu}\nonumber\\
\sum_{\alpha =1}^{N_{SPH}}\nu _{\alpha }\frac{1}{\sigma_{\alpha}}W(|\mathbf{r}-\mathbf{r}_{\alpha }(t)|)\frac{d}{dt}\left(\frac{s}{\sigma}\right)_{\alpha}&=&\sum_{\alpha =1}^{N_{SPH}}\nu _{\alpha }\frac{1}{\sigma_{\alpha}}\frac{1}{T_{\alpha}}\frac{\Pi_{\alpha}}{\sigma^*_{\alpha} }(\partial_{\mu}u^{\mu})_{\alpha}W(|\mathbf{r}-\mathbf{r}_{\alpha }(t)|)\nonumber\\
\frac{d}{dt}\left(\frac{s}{\sigma}\right)_{\alpha}&=&\frac{1}{T_{\alpha}}\frac{\Pi_{\alpha}}{\sigma^*_{\alpha} }(\partial_{\mu}u^{\mu})_{\alpha}
\end{eqnarray}
We also need to be able to express the momentum conservation equation in terms of SPH variables.  Thus, taking Eq.\ (\ref{eqn:idealconsv}) and only the bulk viscosity term from Eq.\ (\ref{eqn:tmunuvisc}) we can write the space compenent in terms of the reference density (the indentity in Eq.\ (\ref{eqn:forcebulk}) was also used)
\begin{eqnarray}
0&=&\partial_{\mu}\left[T^{\mu\nu}_{0}+\Pi^{\mu\nu}\right]\nonumber\\
&=&\partial_{\mu}\left[\epsilon u^{\mu}u^{\nu}-\left(p+\Pi\right)\Delta^{\mu\nu}\right]\nonumber\\
&=&\partial_{\mu}\left[u^{\mu}\left(\epsilon +p+\Pi\right)u^{\nu}\right]-\partial^{\nu}\left(p+\Pi\right)\nonumber\\
&=&\sigma\frac{d}{d\tau}\left(\frac{\epsilon +p+\Pi}{\sigma}u^{\nu}\right)-\partial^{\nu}\left(p+\Pi\right)\nonumber\\
\end{eqnarray}
We can then switch $\nu$ to $i$ because we want to consider only the spatial component so
\begin{equation}\label{eqn:spatcom}
\sigma\frac{d}{d\tau}\left(\frac{\epsilon +p+\Pi}{\sigma}u^{i}\right)=\partial^{i}\left(p+\Pi\right)
\end{equation}
Now in the ideal case the SPH equation of motion can be derived by the variational method.  However, that is not possible when we consider viscosity, thus, we create an equation that returns to the ideal case when the visocisty is zero
\begin{equation}\label{eqn:spheom}
\sigma_{\alpha}\frac{d}{d\tau_{\alpha}}\left(\frac{\epsilon_{\alpha} +p_{\alpha}+\Pi_{\alpha}}{\sigma_{\alpha}}u^{i}_{\alpha}\right)=\sum_{\beta=1}^{N_{SPH}}\nu_{\beta}\sigma_{\alpha}^*\left[\frac{p_{\beta}+\Pi_{\beta}}{\left(\sigma^*_{\beta}\right)^2}+\frac{p_{\alpha}+\Pi_{\alpha}}{\left(\sigma^*_{\alpha}\right)^2}\right]\times\partial^{i}W(|{\bf r}_{\alpha}-{\bf r}_{\beta}(t)|).
\end{equation}
If we take $\Pi=0$ then Eq.\ (\ref{eqn:spheom}) returns the equation of motion for an ideal fluid (see cite{Hama:2004rr})
\begin{equation}
\sigma_{\alpha}\frac{d}{d\tau_{\alpha}}\left(\frac{\epsilon_{\alpha} +p_{\alpha}}{\sigma_{\alpha}}u^{i}_{\alpha}\right)=\sum_{\beta=1}^{N_{SPH}}\nu_{\beta}\sigma_{\alpha}^*\left[\frac{p_{\beta}}{\left(\sigma^*_{\beta}\right)^2}+\frac{p_{\alpha}}{\left(\sigma^*_{\alpha}\right)^2}\right]\times\partial^{i}W(|{\bf r}_{\alpha}-{\bf r}_{\beta}(t)|).
\end{equation}
In order to proceed it is useful introduce the following relations
\begin{eqnarray}
\partial_{\mu}\left(\frac{1}{\sigma}\right)&=&-\frac{1}{\sigma^2}\partial_{\mu}\sigma\label{eqn:sigrel}\\
\partial_{\mu}u^{\mu}&=&-\frac{1}{\sigma}\partial_{\mu}\sigma\label{eqn:sigrel2}\\
\partial^{\mu}u_{\mu}&=&-\frac{\gamma}{\sigma^*}\frac{d\sigma^*}{d\tau}-\frac{g^{ij}u_i}{\gamma}\frac{du_j}{d\tau}\label{eqn:sigrel3}
\end{eqnarray}
as shown in Appendix \ref{app:imptrel}
and we recall that 
\begin{equation}\label{eqn:dervs}
u^{\mu}\partial_{\mu}=\frac{d}{d\tau}=\gamma\frac{d}{dt}.
\end{equation}

Focusing on the left hand side of Eq.\ (\ref{eqn:spheom}) we find
\begin{eqnarray}
\sigma_{\alpha}\frac{d}{d\tau_{\alpha}}\left(\frac{\epsilon_{\alpha} +p_{\alpha}+\Pi_{\alpha}}{\sigma_{\alpha}}u^{i}_{\alpha}\right)&=&\left(\epsilon_{\alpha} +p_{\alpha}+\Pi_{\alpha}\right)\frac{du^{i}_{\alpha}}{d\tau_{\alpha}}+\sigma_{\alpha} u^i_{\alpha} \left(\epsilon_{\alpha} +p_{\alpha}+\Pi_{\alpha}\right) \frac{d}{d\tau_{\alpha}}\left(\frac{1}{\sigma_{\alpha}}\right) +u^i_{\alpha}     \frac{d}{d\tau_{\alpha}}\left(\epsilon_{\alpha} +p_{\alpha}\right) +  u^i_{\alpha}     \frac{d\Pi_{\alpha}}{d\tau_{\alpha}}\nonumber\\
&=&\left(\epsilon_{\alpha} +p_{\alpha}+\Pi_{\alpha}\right)\frac{du^{i}_{\alpha}}{d\tau_{\alpha}}- \frac{u^i_{\alpha}}{\sigma_{\alpha}} \left(\epsilon_{\alpha} +p_{\alpha}+\Pi_{\alpha}\right) \frac{d\sigma_{\alpha}}{d\tau_{\alpha}} +u^i_{\alpha}     \frac{d}{d\tau_{\alpha}}\left(\epsilon_{\alpha} +p_{\alpha}\right) +  u^i_{\alpha}     \frac{d\Pi_{\alpha}}{d\tau_{\alpha}}\nonumber\\
\end{eqnarray}
We can subsitute in Eq.\ (\ref{eqn:mf})
\begin{equation}
\frac{d\tilde{\Pi}}{d\tau}=-\frac{\zeta}{\sigma\tau_R}\partial_{\alpha}u^{\alpha}-\frac{\tilde{\Pi}}{\tau_R}
\end{equation}
so
\begin{eqnarray}
\sigma_{\alpha}\frac{d}{d\tau_{\alpha}}\left(\frac{\epsilon_{\alpha} +p_{\alpha}+\Pi_{\alpha}}{\sigma_{\alpha}}u^{i}_{\alpha}\right)
&=&\left(\epsilon_{\alpha} +p_{\alpha}+\Pi_{\alpha}\right)\frac{du^{i}_{\alpha}}{d\tau_{\alpha}}- \frac{u^i_{\alpha}}{\sigma_{\alpha}} \left(\epsilon_{\alpha} +p_{\alpha}+\Pi_{\alpha}\right) \frac{d\sigma_{\alpha}}{d\tau_{\alpha}} +u^i_{\alpha}     \frac{d}{d\tau_{\alpha}}\left(\epsilon_{\alpha} -p_{\alpha}\right) \nonumber\\
& &-  u^i_{\alpha}    \left(\frac{\zeta_{\alpha}}{\sigma_{\alpha}(\tau_R)_{\alpha}}\partial_{\mu}u^{\mu}_{\alpha}+\frac{\Pi_{\alpha}}{(\tau_R)_{\alpha}}\right)\nonumber\\
&=&\left(\epsilon_{\alpha} +p_{\alpha}+\Pi_{\alpha}\right)\frac{du^{i}_{\alpha}}{d\tau_{\alpha}}- \frac{u^i_{\alpha}}{\sigma_{\alpha}} \left(\epsilon_{\alpha} +p_{\alpha}+\Pi_{\alpha}\right) \frac{d\sigma_{\alpha}}{d\tau_{\alpha}} +u^i_{\alpha}     \frac{d}{ds_{\alpha}}\left(\epsilon_{\alpha} -p_{\alpha}\right)\frac{ds}{d\tau_{\alpha}} \nonumber\\
&& - u^i_{\alpha}     \left(\frac{\zeta_{\alpha}}{\sigma_{\alpha}(\tau_R)_{\alpha}}\partial_{\mu}u^{\mu}_{\alpha}+\frac{\Pi_{\alpha}}{(\tau_R)_{\alpha}}\right).
\end{eqnarray}
Then, using Eq.\ (\ref{eqn:nonconentr})
\begin{eqnarray}\label{eqn:nonconentr}
\partial_{\mu}s^{\mu}&=&-\frac{1}{T}\Pi\partial_{\mu}u^{\mu}\nonumber\\
\partial_{\mu}\left(s u^{\mu}\right)&=&\nonumber\\
u^{\mu}\partial_{\mu}\left(s \right)+s\partial_{\mu}\left( u^{\mu}\right)&=&\nonumber\\
\frac{ds}{d\tau}+s\partial_{\mu}\left( u^{\mu}\right)&=&\nonumber\\
\frac{ds}{d\tau}&=&-\left(\frac{\Pi}{T}+s\right)\partial_{\mu}u^{\mu},
\end{eqnarray}
we find
\begin{eqnarray}
\sigma_{\alpha}\frac{d}{d\tau_{\alpha}}\left(\frac{\epsilon_{\alpha} +p_{\alpha}+\Pi_{\alpha}}{\sigma_{\alpha}}u^{i}_{\alpha}\right)
&=&\left(\epsilon_{\alpha} +p_{\alpha}+\Pi_{\alpha}\right)\frac{du^{i}_{\alpha}}{d\tau_{\alpha}}- \frac{u^i_{\alpha}}{\sigma_{\alpha}} \left(\epsilon_{\alpha} +p_{\alpha}+\Pi_{\alpha}\right) \frac{d\sigma_{\alpha}}{d\tau_{\alpha}} -u^i_{\alpha}     \frac{d}{ds_{\alpha}}\left(\epsilon_{\alpha} -p_{\alpha}\right)\left(\frac{\Pi_{\alpha}}{T_{\alpha}}+s_{\alpha}\right)\partial_{\mu}u^{\mu}_{\alpha} \nonumber\\
&& - u^i_{\alpha}     \left(\frac{\zeta_{\alpha}}{\sigma_{\alpha}(\tau_R)_{\alpha}}\partial_{\mu}u^{\mu}_{\alpha}+\frac{\Pi_{\alpha}}{(\tau_R)_{\alpha}}\right).
\end{eqnarray}
Applying Eq.\ (\ref{eqn:sigrel2}),
\begin{eqnarray}
\sigma_{\alpha}\frac{d}{d\tau_{\alpha}}\left(\frac{\epsilon_{\alpha} +p_{\alpha}+\Pi_{\alpha}}{\sigma_{\alpha}}u^{i}_{\alpha}\right)
&=&\left(\epsilon_{\alpha} +p_{\alpha}+\Pi_{\alpha}\right)\frac{du^{i}_{\alpha}}{d\tau_{\alpha}}- u^i_{\alpha} \left(\epsilon_{\alpha} +p_{\alpha}+\Pi_{\alpha}\right) \partial_{\mu}u^{\mu}_{\alpha} -u^i_{\alpha}     \frac{d}{ds_{\alpha}}\left(\epsilon_{\alpha} -p_{\alpha}\right)\left(\frac{\Pi_{\alpha}}{T_{\alpha}}+s_{\alpha}\right)\partial_{\mu}u^{\mu}_{\alpha} \nonumber\\
&& - u^i_{\alpha}    \left(\frac{\zeta_{\alpha}}{\sigma_{\alpha}(\tau_R)_{\alpha}}\partial_{\mu}u^{\mu}_{\alpha}+\frac{\Pi_{\alpha}}{(\tau_R)_{\alpha}}\right)\nonumber\\
&=&\left(\epsilon_{\alpha} +p_{\alpha}+\Pi_{\alpha}\right)\frac{du^{i}_{\alpha}}{d\tau_{\alpha}}- u^i_{\alpha}    \frac{\Pi_{\alpha}}{(\tau_R)_{\alpha}}\nonumber\\
& &+ u^i_{\alpha}\partial_{\mu}u^{\mu}_{\alpha}  \left[\epsilon_{\alpha} +p_{\alpha}+\Pi_{\alpha} -\frac{d}{ds_{\alpha}}\left(\epsilon_{\alpha} -p_{\alpha}\right)\left(\frac{\Pi_{\alpha}}{T_{\alpha}}+s_{\alpha}\right) -\frac{\zeta_{\alpha}}{\sigma_{\alpha}(\tau_R)_{\alpha}}\right].
\end{eqnarray}
Then we define
\begin{equation}
A_{\alpha}\equiv \epsilon_{\alpha} +p_{\alpha}+\Pi_{\alpha} -\frac{d}{ds_{\alpha}}\left(\epsilon_{\alpha} -p_{\alpha}\right)\left(\frac{\Pi_{\alpha}}{T_{\alpha}}+s_{\alpha}\right) -\frac{\zeta_{\alpha}}{\sigma_{\alpha}(\tau_R)_{\alpha}}
\end{equation}
Then, 
\begin{eqnarray}
\sigma_{\alpha}\frac{d}{d\tau_{\alpha}}\left(\frac{\epsilon_{\alpha} +p_{\alpha}+\Pi_{\alpha}}{\sigma_{\alpha}}u^{i}_{\alpha}\right)
&=&\left(\epsilon_{\alpha} +p_{\alpha}+\Pi_{\alpha}\right)\frac{du^{i}_{\alpha}}{d\tau_{\alpha}}- u^i_{\alpha}    \frac{\Pi_{\alpha}}{(\tau_R)_{\alpha}}+A_{\alpha} u^i_{\alpha}\partial_{\mu}u^{\mu}_{\alpha} \nonumber\\
&=&\left(\epsilon_{\alpha} +p_{\alpha}+\Pi_{\alpha}\right)\frac{du^{i}_{\alpha}}{d\tau_{\alpha}}- u^i_{\alpha}    \frac{\Pi_{\alpha}}{(\tau_R)_{\alpha}}-A_{\alpha} u^i_{\alpha}\frac{u^l_{\alpha} (g_{lm})_{\alpha}}{\gamma_{\alpha}}\frac{d(u^m)_{\alpha}}{d\tau_{\alpha}}-A_{\alpha} u^i_{\alpha}\frac{\gamma_{\alpha}}{\sigma^*_{\alpha}}\frac{d\sigma^*_{\alpha}}{dt_{\alpha}}
\end{eqnarray}
Recalling Eq.\ (\ref{eqn:spatcom}),
\begin{eqnarray}
\partial^{i}\left(p_{\alpha}+\Pi_{\alpha}\right)
&=&\left(\epsilon_{\alpha} +p_{\alpha}+\Pi_{\alpha}\right)\frac{du^{i}_{\alpha}}{d\tau_{\alpha}}- u^i_{\alpha}    \frac{\Pi_{\alpha}}{(\tau_R)_{\alpha}}-A_{\alpha} u^i_{\alpha}\frac{u^l_{\alpha} (g_{lm})_{\alpha}}{\gamma_{\alpha}}\frac{d(u^m)_{\alpha}}{d\tau_{\alpha}}-A_{\alpha} u^i_{\alpha}\frac{\gamma_{\alpha}}{\sigma^*_{\alpha}}\frac{d\sigma^*_{\alpha}}{dt_{\alpha}}\nonumber\\
\underbrace{\left[\left(\epsilon_{\alpha} +p_{\alpha}+\Pi_{\alpha}\right)-\frac{A_{\alpha}}{\gamma_{\alpha}}u^l_{\alpha}u^m_{\alpha} (g_{mi})_{\alpha}\right]}_{=M_{\alpha}^{lmi}}\frac{du^i_{\alpha}}{d\tau_{\alpha}}&=&\underbrace{u^i_{\alpha}    \frac{\Pi_{\alpha}}{(\tau_R)_{\alpha}}+A_{\alpha} u^i_{\alpha}\frac{\gamma_{\alpha}}{\sigma^*_{\alpha}}\frac{d\sigma^*_{\alpha}}{dt_{\alpha}}+\partial^{i}\left(p_{\alpha}+\Pi_{\alpha}\right)}_{F^i_{\alpha}}\label{eqn:forcemass}.
\end{eqnarray}
In Eq.\ (\ref{eqn:forcemass}) we were able to separate the equation into force, mass and accelleration terms such that
\begin{eqnarray}
M_{\alpha}^{lmi}&\equiv&\left(\epsilon_{\alpha} +p_{\alpha}+\Pi_{\alpha}\right)-\frac{A_{\alpha}}{\gamma_{\alpha}}u^l_{\alpha}u^m_{\alpha} (g_{mi})_{\alpha}\\
F^i_{\alpha}&\equiv&u^i_{\alpha}    \frac{\Pi_{\alpha}}{(\tau_R)_{\alpha}}+A_{\alpha} u^i_{\alpha}\frac{\gamma_{\alpha}}{\sigma^*_{\alpha}}\frac{d\sigma^*_{\alpha}}{dt_{\alpha}}+\partial^{i}\left(p_{\alpha}+\Pi_{\alpha}\right),
\end{eqnarray}
which leaves the equation
\begin{equation}
M_{\alpha}^{lmi} \frac{du^i_{\alpha}}{d\tau_{\alpha}}=F^i_{\alpha}
\end{equation}
to solve.

Then we are left with the following coupled differential equations to solve
\begin{eqnarray}
M_{\alpha}^{lmi} \frac{du^i_{\alpha}}{d\tau_{\alpha}}&=&F^i_{\alpha}\\
\gamma _{\alpha }\frac{d\Pi _{\alpha }}{dt}&=&-\frac{\zeta }{(\tau _{R})_{\alpha}}\left(
\partial _{\mu }u^{\mu }\right) _{\alpha }-\frac{1}{(\tau _{R})_{\alpha}}\Pi _{\alpha }\\
\frac{d}{dt}\left(\frac{s }{\sigma}\right) _{\alpha }&=&-\frac{1}{T _{\alpha }}\frac{\Pi _{\alpha }}{\sigma^* _{\alpha } }(\partial_{\mu}u^{\mu}) _{\alpha }
\end{eqnarray}
where we recall that
\begin{equation}
(\partial_{\mu}u^{\mu}) _{\alpha }=-\frac{\gamma _{\alpha }}{\sigma^* _{\alpha }}\frac{d\sigma^* _{\alpha }}{d\tau _{\alpha }}-\frac{(g_{ij} )_{\alpha }u^i _{\alpha }}{\gamma _{\alpha }}\frac{du^j _{\alpha }}{d\tau _{\alpha }}.
\end{equation}

\subsubsection{Hyperbolic Coordinates}





















\appendix
\section{Important relationships}\label{app:imptrel}

In this Appendix we prove Eqs.\ (\ref{eqn:pf1}-\ref{eqn:pf4},\ref{eqn:sigrel}). 
\subsection{$\Delta_{\alpha\beta}\Delta^{\alpha\beta}=3$}
First, we prove Eq.\ (\ref{eqn:pf1}), i.e. $\Delta_{\alpha\beta}\Delta^{\alpha\beta}=3$
\begin{eqnarray}
\Delta_{\alpha\beta}\Delta^{\alpha\beta}&=&\left(g_{\alpha\beta}-u_{\alpha}u_{\beta}\right)\left(g^{\alpha\beta}-u^{\alpha}u^{\beta}\right)\nonumber\\
&=&\underbrace{g_{\alpha\beta}g^{\alpha\beta}}_\textrm{=4}-u_{\alpha}u_{\beta}g^{\alpha\beta}-u^{\alpha}u^{\beta}g_{\alpha\beta}+\underbrace{u_{\alpha}u_{\beta}u^{\alpha}u^{\beta}}_\textrm{=1}\nonumber\\
&=&5-2u^{\alpha}u_{\alpha}\nonumber\\
&=&3															
\end{eqnarray}
Eq.\ (\ref{eqn:pf2}), i.e. $\Delta_{\alpha\beta}\Delta^{\beta\nu}=\Delta_{\alpha}^{\nu}$, is shown by
\begin{eqnarray}
\Delta_{\alpha\beta}\Delta^{\beta\nu}&=&\left(g_{\alpha\beta}-u_{\alpha}u_{\beta}\right)\left(g^{\beta\nu}-u^{\beta}u^{\nu}\right)\nonumber\\
&=&g_{\alpha\beta}g^{\beta\nu}-u_{\alpha}u_{\beta}g^{\beta\nu}-u^{\beta}u^{\nu}g_{\alpha\beta}+u_{\alpha}u_{\beta}u^{\beta}u^{\nu}\nonumber\\	
&=&g_{\alpha}^{\nu}-u_{\alpha}u^{\nu}\nonumber\\	
&=&\Delta_{\alpha}^{\nu}													
\end{eqnarray}

\subsection{$\Delta^{\mu\nu}_{\alpha\beta}\Delta_{\nu\rho}^{\alpha\beta}=\frac{5}{3}\Delta^{\mu}_{\rho}$}

Eq.\ (\ref{eqn:pf2}), i.e. $\Delta^{\mu\nu}_{\alpha\beta}\Delta_{\nu\rho}^{\alpha\beta}=\frac{5}{3}\Delta^{\mu}_{\rho}$, is shown by
\begin{eqnarray}
\Delta^{\mu\nu}_{\alpha\beta}\Delta_{\nu\rho}^{\alpha\beta}&=&\frac{1}{4}\left[\Delta_{\alpha}^{\mu}\Delta_{\beta}^{\nu}+\Delta_{\alpha}^{\nu}\Delta_{\beta}^{\mu}-\frac{2}{3}\Delta^{\mu\nu}\Delta_{\alpha\beta}\right]\left[\Delta^{\alpha}_{\nu}\Delta^{\beta}_{\rho}+\Delta^{\beta}_{\nu}\Delta^{\alpha}_{\rho}-\frac{2}{3}\Delta^{\alpha\beta}\Delta_{\nu\rho}\right]\nonumber\\
&=&\frac{1}{4}\left[\Delta_{\alpha}^{\mu}\Delta_{\beta}^{\nu}\Delta^{\alpha}_{\nu}\Delta^{\beta}_{\rho}+\Delta_{\alpha}^{\mu}\Delta_{\beta}^{\nu}\Delta^{\beta}_{\nu}\Delta^{\alpha}_{\rho}-\frac{2}{3}\Delta_{\alpha}^{\mu}\Delta_{\beta}^{\nu}\Delta^{\alpha\beta}\Delta_{\nu\rho}+\Delta_{\alpha}^{\nu}\Delta_{\beta}^{\mu}\Delta^{\alpha}_{\nu}\Delta^{\beta}_{\rho}+\Delta_{\alpha}^{\nu}\Delta_{\beta}^{\mu}\Delta^{\beta}_{\nu}\Delta^{\alpha}_{\rho}-\frac{2}{3}\Delta_{\alpha}^{\nu}\Delta_{\beta}^{\mu}\Delta^{\alpha\beta}\Delta_{\nu\rho}\right.\nonumber\\
&-&\left.\frac{2}{3}\Delta^{\alpha}_{\nu}\Delta^{\beta}_{\rho}\Delta^{\mu\nu}\Delta_{\alpha\beta}-\frac{2}{3}\Delta^{\beta}_{\nu}\Delta^{\alpha}_{\rho}\Delta^{\mu\nu}\Delta_{\alpha\beta}+\frac{4}{6}\Delta^{\alpha\beta}\Delta_{\nu\rho}\Delta^{\mu\nu}\Delta_{\alpha\beta}\right]\nonumber\\
&=&\frac{1}{4}\left[\Delta^{\mu}_{\rho}+3\Delta^{\mu}_{\rho}-\frac{2}{3}\Delta^{\mu}_{\rho}+3\Delta^{\mu}_{\rho}+\Delta^{\mu}_{\rho}-\frac{2}{3}\Delta_{\rho}^{\mu}-\frac{2}{3}\Delta^{\mu}_{\rho}-\frac{2}{3}\Delta^{\mu}_{\rho}+\frac{4}{9}3\Delta_{\rho}^{\mu}\right]\nonumber\\
&=&\frac{5}{3}\Delta_{\rho}^{\mu}
\end{eqnarray}

\subsection{$\partial^{\mu}-\nabla^{\mu}=u^{\mu}D$}

To find a relationship between $\nabla_{\perp}^{\mu}$, $\partial^{\mu}$, and $u^{\mu}D$, which will be useful when we solve the equations of motion, we prove Eq.\ (\ref{eqn:pf4})
\begin{eqnarray}
\partial^{\mu}-\nabla^{\mu}&=&\partial_{\alpha}g^{\mu\alpha}-\Delta^{\mu\alpha}\partial_{\alpha}\nonumber\\
&=&\left[g^{\mu\alpha}-\Delta^{\mu\alpha}\right]\partial_{\alpha}\nonumber\\
&=&u^{\mu}u^{\alpha}\partial_{\alpha}\nonumber\\
&=&u^{\mu}D
\end{eqnarray}

\subsection{$\partial_{\mu}\left(\frac{1}{\sigma}\right)=-\frac{1}{\sigma^2}\partial_{\mu}\sigma$}

Proving Eq.\ (\ref{eqn:sigrel}), i.e. $\partial_{\mu}\left(\frac{1}{\sigma}\right)=-\frac{1}{\sigma^2}\partial_{\mu}\sigma$,
\begin{eqnarray}
\partial_{\mu}\sigma&=&\partial_{\mu}\left(\sigma^2\frac{1}{\sigma}\right)\nonumber\\
&=&\frac{1}{\sigma}\partial_{\mu}\left(\sigma^2\right)+\sigma^2\partial_{\mu}\left(\frac{1}{\sigma}\right)\nonumber\\
&=&2\partial_{\mu}\sigma+\sigma^2\partial_{\mu}\left(\frac{1}{\sigma}\right)\nonumber\\
\partial_{\mu}\left(\frac{1}{\sigma}\right)&=&  -\frac{1}{\sigma^2}\partial_{\mu}\sigma
\end{eqnarray}
Then, multiplying by $\sigma$ we find
\begin{eqnarray}
\partial_{\mu}u^{\mu}&=&\sigma\partial_{\mu}\left(\frac{1}{\sigma}\right)\nonumber\\
&=&-\frac{1}{\sigma}\partial_{\mu}\sigma
\end{eqnarray}

\subsection{ $\partial^{\mu}u_{\mu}=-\frac{\gamma}{\sigma^*}\frac{d\sigma^*}{d\tau}-\frac{g^{ij}u_i}{\gamma}\frac{du_j}{d\tau}-\frac{u_i u_j}{2\gamma}\frac{d g^{ij}}{d\tau}$}

In order to prove Eq.\ (\ref{eqn:sigrel3}), i.e. $\partial^{\mu}u_{\mu}=-\frac{\gamma}{\sigma^*}\frac{d\sigma^*}{d\tau}-\frac{g^{ij}u_i}{\gamma}\frac{du_j}{d\tau}-\frac{u_i u_j}{2\gamma}\frac{d g^{ij}}{d\tau}$, we start with 

\begin{eqnarray}
\partial^{\mu}u_{\mu}&=&-\frac{1}{\sigma}\frac{d\sigma}{d\tau}\nonumber\\
&=&-\frac{\gamma}{\sigma^*}\frac{d}{d\tau}\left(\frac{\sigma^*}{\gamma}\right)\nonumber\\
&=&-\frac{\gamma}{\sigma^*}\gamma\frac{d}{dt}\left(\frac{\sigma^*}{\gamma}\right)\nonumber\\
&=&-\gamma^2\frac{d}{dt}\left(\frac{1}{\gamma}\right)-\frac{\gamma}{\sigma^*}\frac{d\sigma^*}{dt}\label{eqn:allthemgam}
\end{eqnarray}
However, 
\begin{eqnarray}
\frac{d\gamma}{d\tau} &=&\frac{d}{d\tau} \left(\frac{\gamma^2}{\gamma}\right)\nonumber\\
&=&2\frac{d\gamma}{d\tau}+\gamma^2\frac{d}{d\tau} \left(\frac{1}{\gamma}\right)\nonumber\\
\gamma \frac{d}{d\tau} \left(\frac{1}{\gamma}\right)&=&-\frac{1}{\gamma}\frac{d\gamma}{d\tau}
\end{eqnarray}
so that Eq.\ (\ref{eqn:allthemgam}) becomes
\begin{eqnarray}
\partial^{\mu}u_{\mu}&=&\frac{d\gamma}{dt}-\frac{\gamma}{\sigma^*}\frac{d\sigma^*}{dt}.
\end{eqnarray}
We then need to understand $\frac{d\gamma}{d\tau}$ and to do so we remind ourselves of the properties of $u^{\mu}$
\begin{eqnarray}
u^{\mu}&=&\gamma\left(1,\frac{dr^i}{d\tau}\right)\nonumber\\
\frac{du^{\mu}}{d\tau}&=&\left(\frac{d\gamma}{d\tau},\frac{du^i}{d\tau}\right)
\end{eqnarray}
Then,
\begin{eqnarray}
u_{\mu}u^{\mu}&=&1\nonumber\\
u_{\mu}\frac{u^{\mu}}{d\tau}+u^{\mu}\frac{u_{\mu}}{d\tau}&=&0\nonumber\\
2\gamma \frac{d\gamma}{d\tau}+u_i \frac{du^i}{d\tau}+u^i \frac{du_i}{d\tau}&=&\nonumber\\
2\gamma \frac{d\gamma}{d\tau}+u_i \frac{d(g^{ij}u_j)}{d\tau}+g^{ij}u_j \frac{du_i}{d\tau}&=&\nonumber\\
2\gamma \frac{d\gamma}{d\tau}+u_i g^{ij}\frac{du_j}{d\tau}+u_i u_j\frac{dg^{ij}}{d\tau}+g^{ij}u_j \frac{du_i}{d\tau}&=&\nonumber\\
2\gamma \frac{d\gamma}{d\tau}+2u_i g^{ij}\frac{du_j}{d\tau}+u_i u_j\frac{dg^{ij}}{d\tau}&=&\nonumber\\
 \frac{d\gamma}{d\tau}&=& -\frac{u_i g^{ij}}{\gamma}\frac{du_j}{d\tau}-\frac{u_i u_j}{2\gamma}\frac{dg^{ij}}{d\tau}
\end{eqnarray}
so
\begin{eqnarray}
\partial^{\mu}u_{\mu}&=&-\frac{u_i g^{ij}}{\gamma}\frac{du_j}{d\tau}-\frac{u_i u_j}{2\gamma}\frac{dg^{ij}}{d\tau}-\frac{\gamma}{\sigma^*}\frac{d\sigma^*}{dt}.
\end{eqnarray}
Note that in our current coordinates $g^{\mu\nu}=diag(+,-,-,-)$ that $\frac{dg^{ij}}{d\tau}=0$.  We have left it up until this point for reference when we switch coordinate systems. Thus,
\begin{eqnarray}
\partial^{\mu}u_{\mu}&=&-\frac{u_i g^{ij}}{\gamma}\frac{du_j}{d\tau}-\frac{\gamma}{\sigma^*}\frac{d\sigma^*}{dt}.
\end{eqnarray}


\section{Full Equations}\label{app:full}
In \cite{gabriellatest}, a general derivation of relativistic fluid dynamics from the Botzmann equation is performed where the method of moments is used in a novel way that is consistent with Chapman-Enskog theory.  From kinetic theory, the equation for the shear viscosity assuming zero chemical potential (we do not take into account the baryonic current) is
\begin{eqnarray}\label{eqn:shearall}
\Delta_{\alpha \beta}^{\mu\nu}D\pi^{\alpha\beta}+\frac{\pi^{\mu\nu}}{\tau_{\pi}}=2\frac{\eta}{\tau_{\pi}}\sigma^{\mu\nu}+
2\pi_{\alpha}^{\langle\mu}\omega^{\nu\rangle\alpha}-\delta_{\pi\pi}\pi^{\mu\nu}\Theta&-&\tau_{\pi\pi}\pi_{\alpha}^{\langle\mu}\sigma^{\nu\rangle\alpha}+\lambda_{\pi\Pi}\Pi\sigma^{\mu\nu}+\phi_6\Pi\pi^{\mu\nu}+\phi_7\pi^{\lambda\langle\mu}\pi^{\nu\rangle}_{\lambda}\nonumber\\
+\eta_1\omega_{\lambda}^{\langle\mu}\omega^{\nu\rangle\lambda}+\eta_2\Theta\sigma^{\mu\nu}+\eta_3\sigma^{\lambda\langle\mu}\sigma^{\nu\rangle}_{\lambda}&+&\eta_4\sigma^{\langle\mu}_{\lambda}\omega^{\nu\rangle\lambda}+\eta_6 F^{\langle\mu}F^{\nu\rangle}+\eta_9\nabla^{\langle\mu}_{\perp}F^{\nu\rangle}
\end{eqnarray}
where $F^{\mu}=\nabla_{\perp}^{\mu}p$ and our notation is such that, for instance, $\sigma^{\lambda\langle\mu}\sigma^{\nu\rangle}_{\lambda}=\Delta_{\alpha\beta}^{\mu\nu}\sigma^{\lambda\alpha}\sigma_{\lambda}^{\beta}$ and for the bulk viscosity
\begin{eqnarray}\label{eqn:bulkall}
D\Pi+\frac{\Pi}{\tau_{\Pi}}=-\frac{\zeta}{\tau_{\Pi}}\Theta&+&\delta_{\Pi\Pi}\Pi\Theta+\lambda_{\Pi\pi}\Pi^{\mu\nu}\sigma_{\mu\nu}+\phi_1\Pi^2+\nonumber\\
\phi_3\pi_{\mu\nu}\pi^{\mu\nu}+\omega_{\mu\nu}\omega^{\mu\nu}&+&\zeta_1\sigma_{\mu\nu}\sigma^{\mu\nu}+\zeta_2\Theta^2+\zeta_4 F\cdot F+\zeta_7\nabla_{\mu}F^{\mu}
\end{eqnarray}



\section{Shear and Bulk Viscosity}

In the memory function method \cite{Denicol:2010tr}, which is one of the formulations
of relativistic dissipative hydrodynamics, any irreversible current $J$ is induced by a thermodynamic force F as
\begin{equation}\label{eqn:memfunc}
\tau_{R}u^{\mu}\partial_{\mu}J+\left(1+\tau_R \partial_{\mu}u^{\mu}\right)J=F,
\end{equation}
which one can consider as the generalized version of the Maxwell-Cattaneo law where a new transport coefficent the relaxation time, $\tau_R$, is introduced, which ensures that causality is not violated.  The non-linear term $\tau_R \partial_{\mu}u^{\mu}$  is essential to obtain a stable theory \cite{Denicol:2008ha}.
We can then apply Eq. (\ref{eqn:memfunc}) to the shear and bulk viscosities.  For the shear stress tensor we substitute
\begin{eqnarray}
J&=&\pi^{\mu\nu}\nonumber\\
\tau_R&=&\tau_{\pi}\nonumber\\
F&=&\eta\sigma^{\mu\nu}
\end{eqnarray}
where 
\begin{equation}
\sigma^{\mu\nu}=\nabla^{\langle\mu}u^{\nu\rangle}
\end{equation}
which leads to
\begin{eqnarray}
\tau_{\pi}\Delta^{\mu\nu\lambda\rho}u^{\alpha}\partial_{\alpha}\pi_{\lambda\rho}+\pi^{\mu\nu}&=&\eta\sigma^{\mu\nu}-\tau_{\pi}\pi^{\mu\nu}\Theta
\end{eqnarray}
where $A^{\mu\nu}=\Delta^{\mu\nu\alpha\beta}A_{\alpha\beta}$ and $\Delta^{\mu\nu\alpha\beta}=\frac{1}{2}\left(\Delta^{\mu\alpha}\Delta^{\nu\beta}+\Delta^{\mu\beta}\Delta^{\nu\alpha}-\frac{2}{3}\Delta^{\mu\nu}\Delta^{\alpha\beta}\right)$ is 
the symmetric traceless double projection operator, and the thermodynamic force is $\Theta=\partial_{\mu}u^{\mu}$. 
In the case of the bulk viscous tensor
\begin{eqnarray}
J&=&\Pi\nonumber\\
\tau_R&=&\tau_{\Pi}\nonumber\\
F&=&-\zeta\Sigma
\end{eqnarray}
so that
\begin{eqnarray}
\tau_{\Pi}u^{\alpha}\partial_{\alpha}\Pi+\Pi&=&-\zeta\Theta-\tau_{\Pi}\Pi\Theta.
\end{eqnarray}
The values of the relaxation time coefficients for the shear 
and bulk viscosities, $\tau_\pi$ and $\tau_\Pi$, respectively, have been computed within kinetic theory \cite{Denicol:2010xn}.  


\subsection{Hyperbolic Coordinates}

As in the ideal case, we then switch Eqs.\ (\ref{eqn:shearviscpre}-\ref{eqn:bulkviscpre},\ref{eqn:shearall}-\ref{eqn:bulkall}) into hyperbolic coordinates using Eqs.\ (\ref{eqn:covardev},\ref{eqn:dhyper}). Then, 


\begin{thebibliography}{99}
%\cite{Romatschke:2009im}
\bibitem{Romatschke:2009im}
  P.~Romatschke,
  %``New Developments in Relativistic Viscous Hydrodynamics,''
  Int.\ J.\ Mod.\ Phys.\  {\bf E19}, 1-53 (2010).
  [arXiv:0902.3663 [hep-ph]].
\bibitem{Molnar:2001ux}
  D.~Molnar and M.~Gyulassy,
  %``Saturation of elliptic flow at RHIC: Results from the covariant elastic
  %parton cascade model MPC,''
  Nucl.\ Phys.\  A {\bf 697}, 495 (2002)
  [Erratum-ibid.\  A {\bf 703}, 893 (2002)].
\bibitem{McLerran:1993ni}
  L.~D.~McLerran and R.~Venugopalan,
  %``Computing quark and gluon distribution functions for very large nuclei,''
  Phys.\ Rev.\  D {\bf 49} (1994) 2233.
  %[arXiv:hep-ph/9309289].
  %%CITATION = PHRVA,D49,2233;%%
%\cite{McLerran:1993ka}
\bibitem{McLerran:1993ka}
  L.~D.~McLerran and R.~Venugopalan,
  %``Gluon distribution functions for very large nuclei at small transverse
  %momentum,''
  Phys.\ Rev.\  D {\bf 49} (1994) 3352.
  %[arXiv:hep-ph/9311205].
  %%CITATION = PHRVA,D49,3352;%%
\bibitem{Kharzeev:2002ei}
  D.~Kharzeev, E.~Levin and M.~Nardi,
  %``QCD saturation and deuteron nucleus collisions,''
  Nucl.\ Phys.\  A {\bf 730} (2004) 448
  [Erratum-ibid.\  A {\bf 743} (2004) 329].
  %[arXiv:hep-ph/0212316].
  %%CITATION = NUPHA,A730,448;%%
%\cite{Drescher:2006pi}
\bibitem{Drescher:2006pi}
  H.~J.~Drescher, A.~Dumitru, A.~Hayashigaki and Y.~Nara,
  %``The eccentricity in heavy-ion collisions from color glass condensate
  %initial conditions,''
  Phys.\ Rev.\  C {\bf 74} (2006) 044905.
  %[arXiv:nucl-th/0605012].
  %%CITATION = PHRVA,C74,044905;%%
%\cite{Kharzeev:2002ei}  
%\cite{Kolb:2001qz}
\bibitem{Kolb:2001qz}
  P.~F.~Kolb, U.~W.~Heinz, P.~Huovinen, K.~J.~Eskola and K.~Tuominen,
  %``Centrality dependence of multiplicity, transverse energy, and elliptic
  %flow from hydrodynamics,''
  Nucl.\ Phys.\  A {\bf 696} (2001) 197.
  %[arXiv:hep-ph/0103234].
  %%CITATION = NUPHA,A696,197;%%
\bibitem{Alver:2007qw}
  B.~Alver {\it et al.}  [PHOBOS Collaboration],
  %``Elliptic flow fluctuations in Au+Au collisions at $\sqrt{s_{_{\it NN}}} =$
  %200 GeV,''
  arXiv:nucl-ex/0702036.
  %%CITATION = NUCL-EX/0702036;%%
%\cite{Adams:2003zg}
\bibitem{Adams:2003zg}
  J.~Adams {\it et al.}  [STAR Collaboration],
  %``Azimuthal anisotropy at RHIC: The first and fourth harmonics,''
  Phys.\ Rev.\ Lett.\  {\bf 92}, 062301 (2004).
%\cite{Luzum:2008cw}
\bibitem{Luzum:2008cw}
  M.~Luzum and P.~Romatschke,
  %``Conformal Relativistic Viscous Hydrodynamics: Applications to RHIC results
  %at sqrt(s_NN) = 200 GeV,''
  Phys.\ Rev.\  C {\bf 78}, 034915 (2008)
  [Erratum-ibid.\  C {\bf 79}, 039903 (2009)].
  %%CITATION = PHRVA,C78,034915;%%
\bibitem{NoronhaHostler:2008ju}
  J.~Noronha-Hostler, J.~Noronha and C.~Greiner,
  %``Transport Coefficients of Hadronic Matter near $T_c$,''
  Phys.\ Rev.\ Lett.\  {\bf 103}, 172302 (2009).
  %%CITATION = PRLTA,103,172302;%%  
%\cite{Huovinen:2009yb}
\bibitem{Huovinen:2009yb}
  P.~Huovinen and P.~Petreczky,
  %``QCD Equation of State and Hadron Resonance Gas,''
  Nucl.\ Phys.\  A {\bf 837}, 26 (2010).
  %%CITATION = NUPHA,A837,26;%%
%\cite{Socolowski:2004hw}
\bibitem{Socolowski:2004hw}
  O.~.~J.~Socolowski, F.~Grassi, Y.~Hama and T.~Kodama,
%   ``Fluctuations of the initial conditions and the continuous emission in
  %hydro description of two-pion interferometry,''
  Phys.\ Rev.\ Lett.\  {\bf 93}, 182301 (2004).
  %%CITATION = PRLTA,93,182301;%%
%\cite{Andrade:2006yh}
\bibitem{Andrade:2006yh}
  R.~Andrade, F.~Grassi, Y.~Hama, T.~Kodama, O.~Socolowski, Jr.,
  %``On the necessity to include event-by-event fluctuations in experimental evaluation of elliptical flow,''
  Phys.\ Rev.\ Lett.\  {\bf 97}, 202302 (2006).
%\cite{Hama:2005dz}
\bibitem{Hama:2005dz}
  Y.~Hama, R.~P.~G.~Andrade, F.~Grassi, O.~Socolowski, Jr., T.~Kodama, B.~Tavares, S.~S.~Padula,
  %``3D relativistic hydrodynamic computations using lattice-QCD-inspired equations of state,''
  Nucl.\ Phys.\  {\bf A774}, 169-178 (2006).
%\cite{Anderlik:1998et}
\bibitem{Anderlik:1998et}
  C.~Anderlik, L.~P.~Csernai, F.~Grassi, W.~Greiner, Y.~Hama, T.~Kodama, Z.~I.~Lazar, V.~K.~Magas {\it et al.},
  %``Freezeout in hydrodynamical models,''
  Phys.\ Rev.\  {\bf C59}, 3309-3316 (1999).
 \bibitem{CF} F.\ Cooper and G.\ Frye, Phys.\ Rev.\ D {\bf 10}, 186 (1974).

%\cite{Denicol:2008ha}
\bibitem{Denicol:2008ha}
  G.~S.~Denicol, T.~Kodama, T.~Koide, P.~.Mota,
  %``Stability and Causality in relativistic dissipative hydrodynamics,''
  J.\ Phys.\ G {\bf G35}, 115102 (2008).
  [arXiv:0807.3120 [hep-ph]].

\bibitem{Denicol:2010xn}
  G.~S.~Denicol, T.~Koide and D.~H.~Rischke,
  %``Dissipative relativistic fluid dynamics: a new way to derive the equations
  %of motion from kinetic theory,''
  Phys.\ Rev.\ Lett.\  {\bf 105}, 162501 (2010).
\bibitem{SPH1} L.~B.~Lucy, Astrophys.\ J.\ {\bf 82}, 1013 (1977); J.~J.~Monaghan, Annu.\ Rev.\ Astron.\ Astrophys.\ {\bf 30}, 543 (1992).
C.~E.~Aguiar, T.~Kodama, T.~Osadae, and Y.~Hama, J. Phys. G 27, 75 (2001).
\bibitem{SPHrio}
G.~S.~Denicol, T.~Kodama, T.~Koide, P.~.Mota,
  %``Shock propagation and stability in causal dissipative hydrodynamics,''
  Phys.\ Rev.\  {\bf C78}, 034901 (2008); 
\bibitem{prc2007}
T.~Koide, G.~S.~Denicol, P.~.Mota,T.~Kodama,  
  Phys.\ Rev.\  {\bf C75}, 034909 (2007); 
%\cite{Denicol:2010tr}
\bibitem{Denicol:2010tr}
  G.~S.~Denicol, T.~Kodama, T.~Koide,
  %``The effect of shear and bulk viscosities on elliptic flow,''
  J.\ Phys.\ G {\bf G37}, 094040 (2010).
  [arXiv:1002.2394 [nucl-th]].




\bibitem{Meyer:2007ic}
  H.~B.~Meyer,
  %``A Calculation of the shear viscosity in SU(3) gluodynamics,''
  Phys.\ Rev.\  D {\bf 76}, 101701 (2007); 
  %``A Calculation of the bulk viscosity in SU(3) gluodynamics,''
  Phys.\ Rev.\ Lett.\  {\bf 100}, 162001 (2008).

\bibitem{tomoi} S.~Pu, T.~Koide, and D.~H.~Rischke, Phys.\ Rev.\ D {\bf 81}, 114039 (2010).


%\cite{Huovinen:2008te}
\bibitem{numboltz}
  P.~Huovinen, D.~Molnar,
  %``The Applicability of causal dissipative hydrodynamics to relativistic heavy ion collisions,''
  Phys.\ Rev.\  {\bf C79}, 014906 (2009).
  [arXiv:0808.0953 [nucl-th]]; D.~Molnar, P.~Huovinen,
  %``Applicability of viscous hydrodynamics at RHIC,''
  Nucl.\ Phys.\  {\bf A830}, 475C-478C (2009).
  [arXiv:0907.5014 [nucl-th]]; A.~El, Z.~Xu, C.~Greiner,
  %``Third-order relativistic dissipative hydrodynamics,''
  Phys.\ Rev.\  {\bf C81}, 041901 (2010).
  [arXiv:0907.4500 [hep-ph]].
\bibitem{Denicol:2010br}
  G.~S.~Denicol, X.~-G.~Huang, T.~Koide, D.~H.~Rischke,
  %``Consistency of field-theoretical and kinetic calculations of viscous transport coefficients for a relativistic fluid,''
  
  [arXiv:1003.0780 [hep-th]].

\bibitem{gabriellatest}
G.~S.~Denicol, H.~Niemi, E.~Molnar, D.~H.~Rischke, To appear soon. 

\bibitem{astroSPH}
L.B.~Lucy, A.J. {\bf 82}, 1013 (1977); J.J. Monaghan, Annu. Rev. Astron. Astrophys. {\bf 30}, 543 (1992).
\bibitem{SPHheavyion}
C.E.~Aguiar, T.~Kodama, T.~Osada, and Y.~Hama, J. Phys. {\bf 27}, 75 (2001).

%\cite{Hama:2004rr}
\bibitem{Hama:2004rr}
  Y.~Hama, T.~Kodama, O.~Socolowski, Jr.,
  %``Topics on hydrodynamic model of nucleus-nucleus collisions,''
  Braz.\ J.\ Phys.\  {\bf 35}, 24-51 (2005).
  [hep-ph/0407264].


\end{thebibliography}

\end{document}


